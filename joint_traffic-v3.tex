%\documentclass[10pt, onecolumn]{IEEEtran}
%\documentclass[letterpaper, 10 pt, onecolumn, draftcls]{IEEEtran}
%\documentclass[letterpaper, 10 pt, onecolumn]{IEEEtran}
%\documentclass[10pt, onecolumn, draftcls]{IEEEtran}
\documentclass[3p]{elsarticle}

\journal{Transportation Research Part B}

%\usepackage{lineno,hyperref}
%\modulolinenumbers[5]
%\documentclass[review]{elsarticle}
%\documentclass[10pt, onecolumn]{IEEEtran}
%\textwidth=16.6cm \textheight=21.7cm \oddsidemargin=0.03cm \topmargin=-0.2cm
\renewcommand{\baselinestretch}{1.5}

% The following packages can be found on http:\\www.ctan.org
\usepackage{graphicx}
\usepackage{epsfig} % for postscript graphics files
%\usepackage{fancyhdr}
%\usepackage{mathptmx} % assumes new font selection scheme installed
%\usepackage{times} % assumes new font selection scheme installed
\usepackage{amsmath} % assumes amsmath package installed
\usepackage{amssymb}  % assumes amsmath package installed
\usepackage{mathrsfs}
\usepackage{url}
\usepackage{tabularx,ragged2e,booktabs,caption}
\usepackage{bm}
\usepackage{bbm}
\usepackage{color}
\usepackage{theorem}
\usepackage{caption}
\usepackage{multirow}
%\usepackage{cite}
\usepackage{subcaption}
\usepackage{supertabular}
\usepackage{hyperref}
\usepackage{float}
\usepackage{longtable}
\usepackage{booktabs}
\usepackage{array}
\usepackage{url}
%\usepackage{csvsimple,longtable,booktabs}



% for extrarowheight
\allowdisplaybreaks

\input{mymacros_2cln}
\input{lpsymbols} 
%\input{C:/Users/Yannis/Documents/Private/tex/macros/mymacros_2cln}
%\input{C:/Users/Yannis/Documents/Private/tex/macros/lpsymbols} 
%\input{/home/yannisp/Private/tex/macros/mymacros_2cln}
%\input{/home/yannisp/Private/tex/macros/lpsymbols} 
\newcommand{\pf}{\vskip 5 pt \noindent {\bf{Proof: }}}


\usepackage{algorithm}
\usepackage{algorithmicx,algpseudocode}

\DeclareMathOperator{\VI}{\text{VI}}

\renewcommand{\algorithmicrequire}{\textbf{Input:}}

\makeatletter
\newcommand*{\defeq}{\stackrel{\text{def}}{=}}
\makeatother

\makeatletter
\newcommand{\rmnum}[1]{\romannumeral #1}
\newcommand{\Rmnum}[1]{\expandafter\@slowromancap\romannumeral #1@}
\makeatother



\begin{document}
	
\begin{frontmatter}
	
  \title{Joint Data-Driven Recovery of Travel Time Functions and
    Origin-Destination Demand in Multi-Class Transportation
    Networks\footnote{Research partially supported by the NSF under
      grants CNS-1645681, CCF-1527292 and IIS-1237022 and by the ARO
      under grant W911NF-12-1-0390.}}
	

	\author[mymainaddress]{Jing Zhang}
	\ead{jzh@bu.edu}
	
	\author[mysecondaryaddress]{Ioannis Ch. Paschalidis\corref{mycorrespondingauthor}}
	\cortext[mycorrespondingauthor]{Corresponding author.}
	\ead{yannisp@bu.edu}\ead[url]{http://sites.bu.edu/paschalidis/}
	
	\address[mymainaddress]{Division of Systems Eng., Boston University, Brookline, MA 02446.}
	\address[mysecondaryaddress]{Dept. of Electrical and Computer Eng., Boston University, 8 St. Mary's St., Boston, MA 02215.}


\begin{abstract}
  For both single-class and multi-class transportation networks,
  existing works have tackled the problem of adjusting
  Origin-Destination (OD) demand matrices and recovering travel latency
  cost functions. However, these two types of problems are typically
  treated separately. In this paper, we propose a method to jointly
  recover nonparametric travel latency cost functions and adjust OD
  demand matrices. We formulate the problem as a bilevel optimization
  problem in a multi-class transportation network and develop an
  alternating optimization approach to solve it.  Extensive numerical
  experiments using benchmark networks and a real road network (the
  Eastern Massachusetts highway network), ranging from moderate-sized to
  large-sized, demonstrate the effectiveness and efficiency of our
  method.
\end{abstract}
	
	
\begin{keyword}
  
\end{keyword}
	
\end{frontmatter}

\section{Introduction} \label{sec:intro}
	
Depending on whether we put different weights onto the flows from
different classes of vehicles, we can model a transportation network as
\emph{single-class} \cite{dafermos1969traffic} or \emph{multi-class}
\cite{dafermos1972traffic}. Naturally, we can treat the former as a
special case of the latter. Given a general multi-class transportation
network and assuming all users (drivers) choose routes selfishly, the
network reaches an equilibrium in terms of link flows, known as the {\em
  Wardrop equilibrium} \cite{wardrop1952some}. At the equilibrium,
no single driver can ``benefit'' by rerouting.

Mathematically, for a general multi-class transportation network, given
\emph{travel latency (time) cost functions} (we will also simply say
\emph{cost functions}) with respect to link (road/arc) flows and an
Origin-Destination (OD) flow demand matrix, finding the Wardrop
equilibrium is formulated as the Traffic Assignment Problem (TAP), which
has been extensively studied; see, e.g., \cite{patriksson1994traffic}
and the references therein. The TAP, which we will call the
\emph{forward problem} throughout the paper, can be explicitly
formulated as an optimization problem \cite{dafermos1969traffic} or a
Variational Inequality (VI) problem \cite{dafermos1980traffic}.

In practice, however, the OD demand matrix and the cost function are not
readily available. Transportation engineers have been typically using
surveys to estimate demand and making empirical assumptions for the cost
function (e.g., Bureau of Public Roads cost function
\cite{branston1976link}). With the increasing availability of data,
there is an opportunity to obtain these quantities in a more systematic,
data-driven fashion by formulating appropriate \emph{inverse problems}.
More specifically, given an OD demand matrix and the Wardrop
equilibrium, recovering the cost function has been recently studied for
single-class networks~\cite{bertsimas2014data,CDC16,IFAC17}. Our recent
work \cite{CDC17}, a preliminary version of the current paper, extends
the same type of inverse problem to multi-class models but without
considering them jointly with demand estimation. At the same time,
another version of the inverse problem which adjusts the OD demand
matrix given the Wardrop equilibrium and the cost function has also been
studied for both
single-class~\cite{spiess1990gradient,lundgren2008heuristic,IFAC17,ieee18}
and multi-class~\cite{noriega2007multi} networks. For convenience, in
the sequel, we use the term IP-1 (resp., IP-2) to indicate the inverse
problem of recovering cost functions (resp., adjusting OD demand
matrices).

Most of the existing work typically deals with these two types of
inverse problems separately; a limitation we seek to address in this
paper. There has been some work to estimate OD demand in a congested
network, leading to bilevel formulations and local optimization methods,
see, e.g.,~\cite{florian1995coordinate,yang1995heuristic}. Closer to the
goal of our work, \cite{yang2001simultaneous} considered the
simultaneous estimation of travel cost and OD demand in a stochastic
user equilibrium setting. Yet, these earlier works consider single-class
networks (with the exception of \cite{noriega2007multi}) and do not
attempt to estimate (nonparametrically) the full structure of the travel
cost functions as we do. Rather, they seek to estimate a sensitivity
constant that adjusts how a given travel cost function affects route
choice probabilities.

It is a well-known fact that, for general multi-class transportation
networks, there do not exist reasonable easily verifiable assumptions
about the cost functions to ensure the existence and uniqueness of the
solution to the \emph{forward problem} (i.e., TAP)
\cite{noriega2007algorithmic}. Therefore, for the multi-class inverse
problem IP-1, it would be hard to establish rigorous theoretical results
under mild easy-to-check conditions. In our preliminary work
\cite{CDC17}, an appropriate formulation for the multi-class IP-1 is
provided and numerical experiments are conducted to empirically validate
the proposed solution.

In this paper, operating on multi-class transportation networks, we seek
to jointly investigate the two related inverse problems -- recovering
cost functions (IP-1) in a non-parametric setting and adjusting OD
demand matrices (IP-2). For a given multi-class network, assume that the
Wardrop equilibrium is observed and that an initial ``rough'' OD demand
matrix is available. To solve the joint problem, we leverage a
generalized bilevel optimization problem formulation and propose a
gradient-based alternating optimization approach. The approach
alternates between estimating the cost function (given demand) and
adjusting the OD demand matrix (given the cost function). To validate
its effectiveness and efficiency, we conduct extensive numerical
experiments. In particular, we implement our algorithms over three
benchmark networks (Sioux-Falls, Berlin-Tiergarten, and Anaheim) ranging
from moderate-sized to large-sized and a real large network representing
the Eastern Massachusetts (EMA) highway network.

The rest of the paper is organized as follows. In Sec.~\ref{sec:preli}
we present the single-class and multi-class transportation network
models, formulate the \emph{forward problem} (TAP), specify the form of
the cost functions, and formulate the \emph{inverse problem} IP-1. We
formulate the joint problem and propose algorithms to solve it in
Sec.~\ref{sec:joint}. Numerical results are shown in
Sec.~\ref{sec:num}. Sec.~\ref{sec:conc} concludes the paper. Additional
experimental details for the EMA highway network are contained in the
Appendix.
	
	
\textbf{Notational conventions:} All vectors are column vectors. For
economy of space, we write $\mathbf x = (x_1, \ldots,
x_{\text{dim}(\bx)})$ to denote the column vector $\bx$, where
$\text{dim}(\bx)$ is its dimensionality. $\bx \ge \textbf{0}$ (with
$\textbf{0}$ being the zero vector) denotes that all entries of a vector
$\bx$ are nonnegative. Denote by $\mathbb{R}_+$ the set of all
nonnegative real numbers. We use ``prime'' to denote the transpose of a
matrix or vector. Unless otherwise specified, $\|\cdot\|$ denotes the
$\ell_2$ norm. Let $\left| \mathcal{D} \right|$ denote the cardinality
of a set $\mathcal{D}$, and $\left[\kern-0.15em\left[ \scrD
  \right]\kern-0.15em\right]$ the set $\left\{ {1, \ldots ,\left| \scrD
    \right|} \right\}$. $A \defeq B$ indicates $A$ is defined using $B$.
	
\section{Preliminaries} \label{sec:preli}
	
\subsection{Transportation network model}  \label{sec:mod}
	
Let us review the transportation network models. We note that the same
single-class model to be presented first has also been adopted in
\cite{bertsimas2014data,CDC16,IFAC17}.
	
\subsubsection{Single-class transportation network
  model} \label{sec:sing-mod}
	
Consider a directed graph denoted by $\left( {\scrV, \scrA} \right)$,
where $\scrV$ denotes the set of nodes and $\scrA$ the set of
links. Assume it is strongly connected.  Let $\bN \in {\left\{ {0,1, -
      1} \right\}^{\left| \scrV \right| \times \left| \scrA \right|}}$
be the node-link incidence matrix, and $\textbf{e}_{a}$ the vector with
an entry equal to 1 corresponding to link $a$ and all the other entries
equal to 0.  Let $\bw = \left( {{w_s},{w_t}} \right)$ denote an
Origin-Destination (OD) pair and $\scrW = \left\{ {{\bw_i}:{\bw_i} =
    \left( {{w_{si}},{w_{ti}}} \right), \,i \in \left[\kern-0.15em\left[
        \scrW \right]\kern-0.15em\right]} \right\}$ the set of all OD
pairs.  Denote by ${d^{\bw}} \ge 0$ the amount of the flow demand from
$w_s$ to $w_t$. Let ${\bd^{\bw}} \in {\mathbb{R}^{\left| \scrV
    \right|}}$ be the vector which is all zeros, except for a
$-{d^{\bw}}$ in the coordinate corresponding to node $w_s$ and a
${d^{\bw}}$ in the coordinate corresponding to node $w_t$.  Let $x_a$ be
the total link flow on link $a \in \scrA$ and $\bx$ the vector of these
flows.  Let $\scrF$ be the set of feasible flow vectors defined by
\begin{align}
\begin{array}{l}
	\scrF \defeq \Big\{ \bx:\exists {\bx^{\bw}} \in \mathbb{R}_ +
		^{\left| \scrA \right|} ~\text{s.t.}~\bx =
		\sum\limits_{\bw \in \scrW} {{\bx^{\bw}}},\, \bN{\bx^{\bw}} = {\bd^{\bw}},\,\forall \bw \in \scrW \Big\}, 
\end{array}   \label{feasi}
\end{align}
where $\bx^\bw$ is the flow vector attributed to OD pair $\bw$.
	
In order to formulate appropriate optimization and inverse optimization
problems, we next state the definition of the \textit{Wardrop equilibrium}.
	
\begin{defi}[\cite{patriksson1994traffic}] \label{cdc16-def1} \emph{A
    feasible flow $\bx^* \in \scrF$ is a {\em Wardrop equilibrium}
    if for every OD pair $\bw = \left( {{w_s},{w_t}} \right) \in
    \mathcal{W}$, and any route connecting $(w_s,w_t)$ with positive
    flow in $\bx^*$, the cost of traveling along that route is no
    greater than the cost of traveling along any other route that
    connects $(w_s,w_t)$. }
\end{defi}
	
In the above definition, the cost of traveling along a route is the sum
of the costs of each of its constituent links.
	
\subsubsection{Multi-class transportation network
  model} \label{sec:multi-mod}
	
Denote by $|\tilde \scrU|$ the number of user (vehicle) classes. Let the
original network be $\big( {\tilde {\scrV},\tilde {\scrA}}, \tilde
{\scrW} \big)$, where $\tilde {\scrV} = \big\{ {{v_i}: i \in
  [\kern-0.15em[ \tilde\scrV ]\kern-0.15em]} \big\}$, $\tilde {\scrA} =
\big\{ {{a_{i}}: i \in [\kern-0.15em[ \tilde\scrA ]\kern-0.15em]}
\big\}$, and $\tilde {\scrW} = \big\{ {{\bw_i}:{\bw_i} \defeq \left(
    {{w_{si}},{w_{ti}}} \right),i \in [\kern-0.15em[ \tilde \scrW
  ]\kern-0.15em]} \big\}$.  We borrow the idea of making $|\tilde
\scrU|$ copies of $\big( {\tilde {\scrV},\tilde {\scrA}}, \tilde {\scrW}
\big)$, each corresponding to a single vehicle class, to obtain an
enlarged single-class network \cite{dafermos1972traffic}. In particular,
we construct a single-class network $\left( \scrV, \scrA,\scrW \right)$,
where
\begin{align}
\scrV &= \big\{ {v(i,u): i \in [\kern-0.15em[ \tilde\scrV 
		]\kern-0.15em], u \in [\kern-0.15em[ \tilde\scrU
		]\kern-0.15em]} \big\}, \label{nodes} \\
	\scrA &= \big\{ {a(i,u): i \in [\kern-0.15em[ \tilde\scrA 
		]\kern-0.15em], u \in [\kern-0.15em[ \tilde\scrU 
		]\kern-0.15em]} \big\}, \label{links} \\
	\scrW &= \big\{ \bw(i,u):\bw(i,u) \defeq \left( {w_s(i,u),w_t(i,u)} \right), i \in [\kern-0.15em[ \tilde\scrW 
	]\kern-0.15em], \,u \in [\kern-0.15em[ \tilde\scrU 
	]\kern-0.15em] \big\}. \label{ods}
\end{align}
	
	
We then can derive the node-link incidence matrix $\bN \in {\left\{
    {0,1, - 1} \right\}^{\left| \scrV \right| \times \left| \scrA
    \right|}}$. Note that $\left| \scrV \right| =
|\tilde\scrU||\tilde\scrV|$, and $\left| \scrA \right| =
|\tilde\scrU||\tilde\scrA|$. For $i \in [\kern-0.15em[ \tilde\scrA
]\kern-0.15em],~ u \in [\kern-0.15em[ \tilde\scrU ]\kern-0.15em]$, let
$\textbf{e}_{iu}$ denote the $\left| \scrA \right|$-dimensional vector
with an entry equal to 1 corresponding to link $a(i,u)$ and all the
other entries equal to 0.  Also, for $i \in [\kern-0.15em[ \tilde\scrW
]\kern-0.15em],~ u \in [\kern-0.15em[ \tilde\scrU ]\kern-0.15em]$, we
denote by ${d^{\bw(i,u)}}$ the flow demand of vehicle class $u$ from
node ${w_s(i,u)}$ to node ${w_t(i,u)}$. Then, we define ${\bd^{\bw}} \in
{\mathbb{R}^{\left| \scrV \right|}}$ as for the single-class case.
Accordingly, the set of feasible flow vectors $\scrF$ can be defined as
in \eqref{feasi}.  For convenience, we vectorize a given demand matrix
as $\bg = \big( {{g_{iu}}; \,i \in [\kern-0.15em[ \tilde \scrW
  ]\kern-0.15em],u \in [\kern-0.15em[ \tilde \scrU ]\kern-0.15em]}
\big)$, where for $i \in [\kern-0.15em[ \tilde \scrW ]\kern-0.15em],\,u
\in [\kern-0.15em[ \tilde \scrU ]\kern-0.15em]$, $g_{iu}$ denotes the
flow demand of vehicle class $u$ for OD pair $i$. In addition, we denote
by $\bg_u = \big( {{g_{iu}}; \,i \in [\kern-0.15em[ \tilde \scrW
  ]\kern-0.15em]} \big)$ the demand vector for vehicle class $u \in
[\kern-0.15em[ \tilde \scrU ]\kern-0.15em]$.
	
	
Write a feasible flow vector $\bx \in \scrF$ as $\bx = \big( {x_{iu};
  \,i \in [\kern-0.15em[ \tilde\scrA ]\kern-0.15em], u \in
  [\kern-0.15em[ \tilde\scrU ]\kern-0.15em]} \big)$, where $x_{iu}$
denotes the flow on link $a(i,u)$. Let $\bx_u = \big( {{x_{iu}}; \,i \in
  [\kern-0.15em[ \tilde \scrA ]\kern-0.15em]} \big)$ be the flow vector
for vehicle class $u \in [\kern-0.15em[ \tilde \scrU ]\kern-0.15em]$ and
${\bx_{a_i}} = \big( {x_{iu}};\, u \in [\kern-0.15em[ \tilde\scrU
]\kern-0.15em]\big)$ the flow vector of all vehicle classes
corresponding to the $i$th \emph{physical} link.  We consider the
following cost function:
\begin{align}
	\bt\left( \bx \right) = \big({t_{iu}}(\bx_{a_i}); \,i \in [\kern-0.15em[ \tilde\scrA 
	]\kern-0.15em], u \in [\kern-0.15em[ \tilde\scrU 
	]\kern-0.15em]\big), \label{costf} 
\end{align}
where the cost on a \emph{physical} link does not depend on the flows
elsewhere, but a \emph{physical} link maps to $|\tilde{\scrU}|$
\emph{conceptual} links, each of which corresponds to a vehicle class.
	
	
It is seen that the single-class model is actually a special case of the
multi-class model, and that, formally, the multi-class model can be
treated as an enlarged single-class model. Thus, in the following, we
only need to consider general multi-class models. However, as is well
known, due to coupling of flows from different classes of vehicles, some
of the properties could be very different for a model with $|\tilde
\scrU| > 1$ compared to the single-class case.
	
	
\subsection{Forward problem and inverse problem IP-1}
	
In this section we formulate the forward problem TAP and the inverse
problem IP-1 for the multi-class transportation network $\left( \scrV,
  \scrA,\scrW \right)$ defined in Sec. \ref{sec:multi-mod} (see
\eqref{nodes}-\eqref{ods}).
	
\subsubsection{The forward problem}  \label{sec:forward}
	
As in \cite{CDC16}, here we refer to the Traffic Assignment Problem
(TAP) as the \textit{forward problem}, whose goal is to find the Wardrop
equilibrium for a given transportation network $\left( \scrV,
  \scrA,\scrW \right)$ with a given travel latency cost function $\bt$
and a given OD demand vector $\bg$.  It is a well-known result that TAP
can be formulated as a Variational Inequality (VI) problem
$\text{VI}\left( {\bt,\scrF} \right)$, defined as follows.
	
\begin{defi}
  \label{cdc16-def2} \em{(\cite{bertsimas2014data})}. The VI problem,
  denoted as $\text{VI}\left( {\bt,\scrF} \right)$, is to find an ${\bx^
    * } \in \scrF$ s.t.
\begin{align}
		\bt{\left( {{\bx^ * }} \right)'}\left( {\bx - {\bx^ * }} \right) \geq 0, \quad \forall \bx \in \scrF. \label{cdc16-VI}
\end{align}
\end{defi}
	
To proceed, let us first present a definition regarding the monotonicity
of a cost function.
\begin{defi}
\label{journal-def3} \em{(\cite{patriksson1994traffic})}. 
		$\bt(\cdot)$ is \textit{strongly monotone} on $\scrF$ if there exists a constant $\eta > 0$ such that
		\[\left( {\bt\left( \bx \right) - \bt\left( \by \right)} \right)'\left( {\bx - \by} \right) \ge \eta {\left\| {\bx - \by} \right\|^2},\quad \forall \bx,\by \in \scrF.\]
\end{defi}
To ensure the existence and uniqueness of the solution to
$\text{VI}\left( {\bt,\scrF} \right)$, we need the following assumption.
\begin{ass} \label{cdc16-assumption1} $\bt(\cdot)$ is strongly monotone
  on $\scrF$ and continuously differentiable on $\mathbb{R}_ + ^{\left|
      \scrA \right|}$. $\scrF$ is nonempty and contains an interior
  point (Slater's condition \cite{boyd2004convex}).
\end{ass}
	
	
Relating the Wardrop equilibrium to the VI problem, the following result
is well-known in the literature.
\begin{thm} \label{cdc17-th21} (\cite{patriksson1994traffic}).
  \noindent Under Assump. \ref{cdc16-assumption1}, a Wardrop equilibrium
  of the multi-class transportation network is a solution to
  $\text{VI}(\bt,\scrF)$, where $\bt,\scrF$ are given by \eqref{costf}
  and \eqref{feasi}, respectively.
\end{thm}
	
\begin{rmk} \label{rem1} \em{As noted in \cite{noriega2007algorithmic},
    Assump. \ref{cdc16-assumption1} cannot be easily verified for
    general multi-class transportation networks. We therefore do not
    have any guarantee of always obtaining unique link flows for each
    and every class of vehicles.}
\end{rmk}
	
	
	
\subsubsection{BPR-type cost functions} \label{sec:cost}
	
To simplify the analysis, we now further specify the cost functions in
\eqref{costf}.  For each $i \in [\kern-0.15em[ \tilde\scrA
]\kern-0.15em], u \in [\kern-0.15em[ \tilde\scrU ]\kern-0.15em]$, we
define the following generalized Bureau of Public Roads (BPR)-type
travel time (latency cost) function \cite{branston1976link,
  noriega2007algorithmic, bertsimas2014data}:
\begin{align}{t_{iu}}\left( \bx \right) = {t^{{0}}_{iu}}f\Big( {\frac{{{\btheta}'{{\bx_{a_i}}} }}{{{m_{i}}}}} \Big), \label{costMulti}
\end{align}
where ${t^{{0}}_{iu}}$ is called the \textit{free-flow travel time} for
vehicle class $u$ on link ${a_i}$ (the $i$th \emph{physical} link),
$f(\cdot)$ is a ``cornerstone'' cost function, which satisfies $f(0)=1$
and is strictly increasing and continuously differentiable on
$\mathbb{R}_+$, ${m_{i}}$ is the \textit{effective flow capacity} of
link ${a_i}$, and ${\btheta} = \big( {\theta _u}; \, u \in
[\kern-0.15em[ \tilde\scrU ]\kern-0.15em] \big)$ is a weight vector such
that ${\theta _u} \geq 1, \,\forall u \in [\kern-0.15em[ \tilde\scrU
]\kern-0.15em]$. As a special case, the single-class network corresponds
to $|\tilde\scrU| = 1$ and $\theta_1 = 1$.
	
	
	
\subsubsection{The inverse problem IP-1} \label{sec InverseVI-multi}
	
In order to solve the forward problem, thus obtaining the Wardrop
equilibrium flow vector $\bx$, we need to know the cost function $\bt$
and the OD demand vector $\bg$. Now, assuming that we know the OD demand
vector $\bg$ and have observed the Wardrop equilibrium flow vector
$\bx$, we seek to formulate the inverse problem IP-1 (as an inverse VI
problem, in particular), so as to estimate the travel latency cost
function $\bt$ (specifically, $f(\cdot)$ in \eqref{costMulti}).  To
provide some insight, given $|\scrK|$ samples of the link flow vector
$\bx$, one can think of them as flow observations on $|\scrK|$ different
networks (or subnetworks) which are nevertheless produced by the exact
same cost function. The inverse problem formulation seeks to determine
the cost function so that each flow observation is as close to an
equilibrium as possible.
	
To that end, for a given $\epsilon > 0$, we define an
\textit{$\epsilon$-approximate solution} to $\VI(\mathbf{t}, \scrF)$ by
changing the right-hand side of \eqref{cdc16-VI} to $- \epsilon$.
\begin{defi}
  \label{cdc16-def3}
  \em{(\cite{bertsimas2014data})}. Given $\epsilon > 0$, $\hat{\bx} \in
  \scrF$ is called an \textit{$\epsilon$-approximate solution} to
  $\VI(\mathbf{t}, \scrF)$ if
\begin{equation} \label{cdc16-eq:defApproxEquil}
\mathbf{t}(\hat{\bx})'(\bx - \hat{\bx}) \geq - \epsilon, \quad
\forall \bx \in \scrF.
\end{equation}
\end{defi}
	
	
Assume now we are given $|\scrK|$ networks
$(\scrV^{{(k)}},\scrA^{{(k)}},\scrW^{{(k)}}),\, k \in [\kern-0.15em[
\scrK ]\kern-0.15em]$ (as a special case, these could be $|\scrK|$
replicas of the same network $\left( {\scrV, \scrA, \scrW} \right)$),
and the observed link flow data $\big( {\bx_{a_i}^{{(k)}}} = \big(
{x^{{(k)}}_{iu}};\, u \in [\kern-0.15em[ \tilde\scrU
]\kern-0.15em]\big); \, i \in [\kern-0.15em[ \tilde\scrA^{{(k)}}
]\kern-0.15em], k \in [\kern-0.15em[ \scrK ]\kern-0.15em]\big)$. The
inverse VI problem amounts to finding a function $\bt$ such that
$\bx^{{(k)}}$ is an $\epsilon_k$-approximate solution to $\VI(\bt,
\scrF^{{(k)}})$ for each $k$.  Denoting $\bepsilon \defeq (\epsilon_k;
\, k \in [\kern-0.15em[ \scrK ]\kern-0.15em])$, we can formulate the
inverse VI problem as in \cite{bertsimas2014data}
\begin{align}
	\min_{{\mathbf{t}}, \bepsilon} \quad & \| \bepsilon \|
	\label{cdc16-inverseVI1} \\
	\text{s.t.} \quad & {\mb{t}}({\bx^{{(k)}}})'(\bx-{\bx^{{(k)}}})
        \geq -\epsilon_k, \quad \forall \bx\in \scrF^{{(k)}}, k \in
        [\kern-0.15em[ \scrK
	]\kern-0.15em], \notag \\
	& \epsilon_k > 0, \quad \forall k \in [\kern-0.15em[ \scrK
        ]\kern-0.15em], \notag
\end{align}
where the optimization is over $\bepsilon$ and the selection of function
$\bt(\cdot)$ (that is, $f(\cdot)$).
	
Aiming at recovering a cost function that has both good data reconciling
and generalization properties, we apply an estimation approach which
expresses the function $f(\cdot)$ (recall \eqref{costMulti}) in a
Reproducing Kernel Hilbert Space (RKHS) $\scrH$
\cite{bertsimas2014data,evgeniou2000regularization}. In particular, by
\cite[Thm. 2]{bertsimas2014data}, being a variant of \cite[(6)]{IFAC17},
the inverse VI problem \eqref{cdc16-inverseVI1} can be reformulated as a
Quadratic Program (QP):
\begin{align}
	\text{(invVI-1)} \notag \\
	\mathop {\min }\limits_{\by,\bepsilon} {\text{  }}& \|\bepsilon\| + \gamma\| {f} \|^2_{\scrH} \label{inverVI2-multi-pre} \\
	\text{s.t.}{\text{  }}&{\textbf{e}}_{iu}'\bN'_k{\by^\bw} \leq t_{iu}^{{0}}{{{f}\bigg( \frac{{{{\btheta}}'\bx_{a_i}^{{(k)}}}}{{{m_{i}^{{(k)}}}}} \bigg)}}, \qquad \forall i \in [\kern-0.15em[ \tilde\scrA^{{(k)}} 
	]\kern-0.15em], ~u \in [\kern-0.15em[ \tilde\scrU 
	]\kern-0.15em], ~\bw \in {\scrW^{{(k)}}}, ~k \in [\kern-0.15em[ \scrK 
	]\kern-0.15em],  \label{dual-feasi} \\
	&\sum\limits_{i=1}^{|\tilde\scrA^{{(k)}}|} \bigg(\sum\limits_{u=1}^{|\tilde\scrU|} {t_{iu}^{{0}}{x_{iu}}}\bigg) {{{f}\bigg( \frac{{{{\btheta}}'\bx_{a_i}^{{(k)}}}}{{{m_{i}^{{(k)}}}}} \bigg)}} - \sum\limits_{\bw \in {\scrW_k}} {\left( {{\bd^\bw}} \right)'{\by^\bw}}  \leq {\epsilon_k},  \qquad \forall k \in [\kern-0.15em[ \scrK 
	]\kern-0.15em],  \label{sub-opt} \\
	&{ {{f}\bigg( \frac{{{{\btheta}}'\bx_{a_i}^{{(k)}}}}{{{m_{i}^{{(k)}}}}} \bigg)}} < {{{f}\bigg( \frac{{{{\btheta}}'\bx_{a_{\tilde i}}^{{(k)}}}}{{{m^{{(k)}}_{{\tilde i}}}}} \bigg)}}, \qquad \forall i, \, \tilde{i} \in [\kern-0.15em[ \tilde\scrA^{{(k)}} 
	]\kern-0.15em]
	~{\text{ s}}{\text{.t}}{\text{. }}\frac{{{{\btheta}}'\bx_{a_i}^{{(k)}}}}{{{m_{i}^{{(k)}}}}} < \frac{{{{\btheta}}'\bx_{a_{\tilde i}}^{{(k)}}}}{{{m^{{(k)}}_{{\tilde i}}}}}; \,\forall k \in [\kern-0.15em[ \scrK 
	]\kern-0.15em],  \label{monoto} \\
	& \bepsilon \geq \textbf{0}, \quad 
	{f} \in \scrH, \notag \\
	&{f(0)} = 1, \label{normaliz} 
\end{align}
where $\bN_k$ is the node-link incidence matrix of the $k$-th network,
$\by = \big( {{\by^{\bw} \in {\mathbb{R}^{| \scrV^{{(k)}} |}}};\, \bw
  \in \scrW^{{(k)}}, k \in [\kern-0.15em[ \scrK ]\kern-0.15em]} \big)$,
$\bepsilon = (\epsilon_k; \,k \in [\kern-0.15em[ \scrK ]\kern-0.15em])$
are decision vectors, $\by^{\bw}$ is a dual variable which can be
interpreted as the ``price'' of $\bd^{\bw}$, $\gamma > 0$ is a
regularization parameter, $\| {f} \|^2_{\scrH}$ denotes the squared norm
of $f(\cdot)$ in $\scrH$, \eqref{dual-feasi} is for dual feasibility,
\eqref{sub-opt} is the suboptimality (primal-dual gap) constraint,
\eqref{monoto} enforces $f(\cdot)$ to be increasing, and
\eqref{normaliz} is for normalization purposes (see \eqref{costMulti}).
Note that a smaller $\gamma$ should result in recovering a ``tighter''
$f(\cdot)$ in terms of data reconciliation whereas a bigger $\gamma$
would lead to a ``better'' $f(\cdot)$ in terms of generalization
properties.
	
It can be seen that the above formulation is still too abstract for us
to solve, because it is an optimization over functions. To make it
tractable, in the following, we specify $\scrH$ by picking its
\textit{reproducing kernel} \cite{evgeniou2000regularization} as a
polynomial ${\phi}(x, y) \defeq ( c + xy )^n$ for some choice of $c \geq
0$ and $n \in \mathbb{N}$. Then, writing
	\[\phi\left( {x,y} \right) = {\left( {c + xy} \right)^n} = \sum\limits_{j = 0}^n {{n \choose j}{c^{n - j}}{x^j}{y^j}}, \]
	by \cite[(3.2), (3.3), and (3.6)]{evgeniou2000regularization},
	we instantiate invVI-1 as \cite{CDC17} 
\begin{align}
	\text{(invVI-2)}  \notag \\
	\mathop {\min }\limits_{\bbeta ,\by,\bepsilon} {\text{  }}& \|\bepsilon\| + \gamma \sum\limits_{j = 0}^n {\frac{{\beta _j^2}}{{{n \choose j}{c^{n - j}}}}} \label{inverVI2-multi} \\
	\text{s.t.}{\text{  }}&{\textbf{e}}_{iu}'\bN'_k{\by^\bw} \leq t_{iu}^{{0}}{\sum\limits_{j = 0}^n {{\beta _j}\bigg( \frac{{{{\btheta}}'\bx_{a_i}^{{(k)}}}}{{{m_{i}^{{(k)}}}}} \bigg)} ^j}, \qquad \forall i \in [\kern-0.15em[ \tilde\scrA^{{(k)}} 
	]\kern-0.15em], ~u \in [\kern-0.15em[ \tilde\scrU 
	]\kern-0.15em], ~\bw \in {\scrW^{{(k)}}}, ~k \in [\kern-0.15em[ \scrK 
	]\kern-0.15em],  \notag \\
	&\sum\limits_{i=1}^{|\tilde\scrA^{{(k)}}|} {\bigg({\sum\limits_{j = 0}^n {{\beta _j}\bigg( \frac{{{{\btheta}}'\bx_{a_i}^{{(k)}}}}{{{m_{i}^{{(k)}}}}} \bigg)} ^j}\bigg)} \sum\limits_{u=1}^{|\tilde\scrU|} {t_{iu}^{{0}}{x_{iu}}} - \sum\limits_{\bw \in {\scrW_k}} {\left( {{\bd^\bw}} \right)'{\by^\bw}}  \leq {\epsilon_k},  \qquad \forall k \in [\kern-0.15em[ \scrK 
	]\kern-0.15em],  \notag \\
	&{\sum\limits_{j = 0}^n {{\beta _j}\bigg( \frac{{{{\btheta}}'\bx_{a_i}^{{(k)}}}}{{{m_{i}^{{(k)}}}}} \bigg)} ^j} < {\sum\limits_{j = 0}^n {{\beta _j}\bigg( \frac{{{{\btheta}}'\bx_{a_{\tilde i}}^{{(k)}}}}{{{m^{{(k)}}_{{\tilde i}}}}} \bigg)} ^j}, \qquad \forall i, \, \tilde{i} \in [\kern-0.15em[ \tilde\scrA^{{(k)}} 
	]\kern-0.15em]
	~{\text{ s}}{\text{.t}}{\text{. }}\frac{{{{\btheta}}'\bx_{a_i}^{{(k)}}}}{{{m_{i}^{{(k)}}}}} < \frac{{{{\btheta}}'\bx_{a_{\tilde i}}^{{(k)}}}}{{{m^{{(k)}}_{{\tilde i}}}}}; \,\forall k \in [\kern-0.15em[ \scrK 
	]\kern-0.15em],  \notag \\
	& \bepsilon \geq \textbf{0}, \notag \\
	&{\beta _0} = 1, \notag 
\end{align}
where the function $f(\cdot)$ in invVI-1 is parametrized by $\bbeta =
\left( {{\beta _i};\;i = 0, 1, \ldots ,n} \right)$.  Assuming an optimal
${\bbeta ^*} = \left( {\beta _j^*;\,j = 0,1, \ldots ,n} \right)$ is
obtained by solving \eqref{inverVI2-multi}, then our estimator for the
cost function $f(\cdot)$ is
\begin{align}
	\hat f\left( x \right) = \sum\limits_{j = 0}^n {\beta _j^*{x^j}}  = 1 + \sum\limits_{j = 1}^n {\beta _j^*{x^j}}.  \label{costEstimator}
\end{align}
	
It is worth pointing out that, in the above QP formulations, the
parameter vector ${\btheta}$ and the set of vehicle classes
$\tilde{\scrU}$ have been assumed to be the same for all $|\scrK|$
networks. We note that in \eqref{costMulti} what essentially gets
involved is only the weighed sum of link flows from different classes of
vehicles (other than the link flow of each single vehicle class). It
turns out that, similar to our previous work on single-class networks
\cite{CDC16,IFAC17}, we are still very likely to be able to recover the
cost functions with satisfactory accuracy from such weighted sum of link
flows.

\section{The Joint Problem} \label{sec:joint}
	
Different from previous works, in this section we solve a joint problem
to recover the cost function $f(\cdot)$ (specifically, the coefficient
vector $\bbeta = \left( {{\beta _j};\,j = 0, 1, \ldots ,n} \right)$ with
$\beta_0 = 1$; cf. \eqref{costEstimator}), hence $\bt$, and adjust the
OD demand vector $\bg$.  To simplify notation, for any given feasible
$\bbeta$ (determining $f(\cdot)$, hence $\bt$) and $\bg$, let
$\bx\left(\bbeta, \bg \right) = \big( {x_{iu}}\left(\bbeta, \bg \right);
\, i \in [\kern-0.15em[ \tilde \scrA ]\kern-0.15em],u \in [\kern-0.15em[
\tilde \scrU ]\kern-0.15em]\big)$ be the optimal solution to
$\text{VI}(\bt,\scrF)$.
	
Assume that we have observed an equilibrium flow vector, denoted by
$\bx^* = \big( {x^*_{iu}; \,i \in [\kern-0.15em[ \tilde\scrA
  ]\kern-0.15em], u \in [\kern-0.15em[ \tilde\scrU ]\kern-0.15em]}
\big)$, and that an initial demand vector ${\bg^0}$ is available. We
seek to solve the following bi-level optimization problem, which is a
variant of \cite[(1)-(2)]{spiess1990gradient},
\cite[(1)-(8)]{noriega2007multi}, and \cite[(9)]{IFAC17}:
\begin{align}
	\text{(BiLev)} \quad \mathop {\min }\limits {\text{  }}&F\left(\bbeta, \bg \right) \defeq
	{\gamma _1}\sum\limits_{i = 1} ^{ |\tilde \scrW| }\sum\limits_{u = 1}^{|\tilde \scrU|} {{{\left( {{g_{iu}} - g^0_{iu}} \right)}^2}} + \gamma_2\sum\limits_{\check{i} = 1}^{|\tilde \scrA|} {\sum\limits_{\check{u} = 1}^{|\tilde \scrU|}{{{\left( {{x_{\check{i}\check{u}}}\left(\bbeta, \bg \right) - {{x}^*_{\check{i}\check{u}}}} \right)}^2}}} \label{bilevel}  \\
	\text{s.t.}\quad & {\bbeta \geq \textbf{0},\,\beta_0 = 1, \,\bg \geq \textbf{0}},  \notag
\end{align}
where $\gamma_1, \gamma_2 \geq 0$ are two weight parameters; the first
term penalizes moving too far away from the initial demand, and the
second term ensures that the optimal solution to TAP is close to the
flow observation. Note that BiLev \eqref{bilevel} is more general than
its counterparts (IP-2, in particular) considered in
\cite{spiess1990gradient,noriega2007multi,IFAC17}.  It is also worth
pointing out that $F\left(\bbeta, \bg \right)$ is bounded below by $0$
which guarantees the convergence of the algorithm (see
Alg.~\ref{alg:joint}) that we will apply.
		
To solve BiLev numerically, thus alternatively recovering $\bbeta$ and
adjusting $\bg$ in an iterative manner, we leverage a gradient-based
algorithm (Alg.~\ref{alg:joint}). Let us first derive an approximation
to the gradient of $F(\bbeta,\bg)$ (defined in \eqref{bilevel}) with
respect to $\bg$. To that end, fix $\check{i} \in [\kern-0.15em[ \tilde
\scrA ]\kern-0.15em], ~\check{u} \in [\kern-0.15em[ \tilde \scrU
]\kern-0.15em]$, and a feasible pair $(\bbeta, \bg)$.  By flow
conservation, we have
\begin{align}
	{x_{\check{i}\check{u}}}\left( {\bbeta ,\bg} \right) = \sum\limits_{i \in [\kern-0.15em[ \tilde \scrW 
			]\kern-0.15em]} {\sum\limits_{r \in {\scrR^i_{\check u}}}} {\delta_r^{a(\check{i},\check{u})} p_{\check{u}}^{ir}g_{i\check{u}} }  = \sum\limits_{i \in [\kern-0.15em[ \tilde \scrW 
			]\kern-0.15em]}g_{i\check{u}} {\sum\limits_{r \in {\scrR^i_{\check u}}}} {\delta_r^{a(\check{i},\check{u})} p_{\check{u}}^{ir} },   \label{linkflow}
\end{align}
where ${\scrR^i_{\check u}}$ denotes the set of feasible routes
associated with OD pair $i$ and user class $\check u$,
$p_{\check{u}}^{ir}$ is the probability for user class $\check{u}$ to
select route $r \in {\scrR^i_{\check u}}$, and
\begin{align}
	\delta_r^{a(\check{i},\check{u})} \defeq \left\{ \begin{gathered}
	1,{\text{ if route }} r \text{ uses link } a(\check{i},\check{u}), \hfill \\
	0,{\text{ otherwise}}{\text{.}} \hfill \\  
	\end{gathered}  \right.   \label{delta}
	\end{align}
	Assuming that the route probabilities are locally constant, then \eqref{linkflow} implies \cite{noriega2007multi, spiess1990gradient}
		\begin{align}
		\frac{{\partial {x_{\check i \check u}}\left(\bbeta, \bg  \right)}}{{\partial {g _{iu}}}} = \left\{ \begin{gathered}
		{\sum\limits_{r \in {\scrR^i_{\check u}}}} {\delta_r^{a(\check{i},\check{u})} p_{\check{u}}^{ir} },{\text{ if }} \check{u} = u, \hfill \\
		0,{\text{ otherwise}}{\text{,}} \hfill \\  
		\end{gathered}  \right. 
		\qquad \forall {i} \in [\kern-0.15em[ \tilde \scrW 
		]\kern-0.15em], ~{u} \in [\kern-0.15em[ \tilde \scrU 
		]\kern-0.15em].  \label{jacobtmp}
\end{align}
Further, for OD pair $i \in [\kern-0.15em[ \tilde \scrW 
	]\kern-0.15em]$ and vehicle class $u \in [\kern-0.15em[ \tilde \scrU 
	]\kern-0.15em]$, considering only the shortest route $r_{iu}(\bbeta,\bg)$ based on the travel latency cost (i.e., travel time), we have 
\begin{align}
	\frac{{\partial {x_{\check i \check u}}\left(\bbeta, \bg  \right)}}{{\partial {g _{iu}}}} \approx {\delta _{{r_{iu}(\bbeta,\bg)}}^{a(\check{i},\check{u})}} &= \left\{ \begin{gathered}
	1,{\text{ if }}a(\check{i},\check{u}) \in {r_{iu}(\bbeta,\bg)}, \hfill \\
	0,{\text{ otherwise}}{\text{,}} \hfill \\  
	\end{gathered}  \right.  \notag \\ 
	&= \left\{ \begin{gathered}
	1,{\text{ if }} \check{u} = u \text{ and } a(\check{i},\check{u}) \in {r_{iu}(\bbeta,\bg)}, \hfill \\
	0,{\text{ otherwise}}{\text{,}} \hfill \\  
	\end{gathered}  \right.  \label{jacob}
\end{align}
where $a(\check{i},\check{u}) \in {r_{iu}(\bbeta,\bg)}$ indicates that
the route $r_{iu}(\bbeta,\bg)$ uses link $a(\check{i},\check{u})$. Note
that $a(\check{i},\check{u}) \notin {r_{iu}(\bbeta,\bg)}$ for all $
\check u \neq u$. Note also that we have assumed the partial derivatives
do exist; if not, one can replace them with subgradients. Such partial
derivatives as in the BiLev setting typically do not have an exact
analytical expression and we in turn use a certain approximation
technique; a comprehensive discussion on the existence and approximation
of such type of partial derivatives can be found in
\cite{patriksson2004sensitivity}. By \eqref{jacob} we obtain an
approximation to the Jacobian matrix
\begin{align}
	\left[ {\frac{{\partial {x_{\check{i}\check{u}}}\left(\bbeta, {{\bg}} \right)}}{{\partial {g _{iu}}}};\, \check{i} \in [\kern-0.15em[ \tilde \scrA 
		]\kern-0.15em],\,\check{u} \in [\kern-0.15em[ \tilde \scrU 
		]\kern-0.15em],\,i \in [\kern-0.15em[ \tilde \scrW 
		]\kern-0.15em], \, u \in [\kern-0.15em[ \tilde \scrU 
		]\kern-0.15em]} \right]. \label{jacobMat}
\end{align}
Let us now compute the gradient of $F\left(\bbeta, \bg \right)$ with
respect to $\bg$. We have
\begin{align}
	\nabla_{\bg} F\left(\bbeta, \bg  \right) &= \left( {\frac{{\partial F\left(\bbeta, \bg  \right)}}{{\partial {g _{iu}}}}};\, i \in [\kern-0.15em[ \tilde \scrW 
	]\kern-0.15em], \, u \in [\kern-0.15em[ \tilde \scrU 
	]\kern-0.15em] \right) \notag \\
	&= \bigg( 2 \gamma_1 \left( {{g_{iu}} - g^0_{iu}} \right) + 2 \gamma_2 {\sum\limits^{|\tilde \scrA|}_{\check{i} = 1} {\sum\limits^{|\tilde \scrU|}_{\check u = 1} {\left( {{x_{\check{i}\check{u}}}\left(\bbeta, \bg  \right) - {{ x}^*_{\check{i}\check{u}}}} \right)\frac{{\partial {x_{\check{i}\check{u}}}\left(\bbeta, \bg  \right)}}{{\partial {g_{iu}}}}} ;\,i \in [\kern-0.15em[ \tilde \scrW 
			]\kern-0.15em], \, u \in [\kern-0.15em[ \tilde \scrU 
			]\kern-0.15em]}} \bigg) \notag \\
	&= \bigg(2 \gamma_1 \left( {{g_{iu}} - g^0_{iu}} \right) + 2 \gamma_2 {\sum\limits^{|\tilde \scrA|}_{\check{i} = 1} {{\left( {{x_{\check{i}u}}\left(\bbeta, \bg  \right) - {{x}^*_{\check{i}u}}} \right)\frac{{\partial {x_{\check{i}u}}\left(\bbeta, \bg  \right)}}{{\partial {g_{iu}}}}} ;\,i \in [\kern-0.15em[ \tilde \scrW 
			]\kern-0.15em], \, u \in [\kern-0.15em[ \tilde \scrU 
			]\kern-0.15em]}} \bigg). \label{gradient}
\end{align}

	
\begin{rmk} \label{rem:shortest} \em{Similar to \cite{IFAC17,ieee18},
    the reasons why we consider only the shortest routes for the purpose
    of calculating the Jacobian include: (\rmnum{1}) GPS navigation is
    widely-used by vehicle drivers so that they tend to always select
    the shortest routes between their OD pairs. (\rmnum{2}) Considering
    the shortest routes only would significantly simplify the
    calculation of the route-choice probabilities. (\rmnum{3}) Extensive
    numerical experiments show that such an approximation as
    \eqref{jacob} for the Jacobian matrix performs satisfactorily well.}
\end{rmk}

We summarize the procedures for alternatively recovering the cost
function and adjusting the OD demand vector as Alg.~\ref{alg:joint},
whose convergence will be proven in the following proposition.
	
\begin{algorithm}
\caption{Alternatively recovering the travel latency cost function and
  adjusting the OD demand vector.}
\label{alg:joint}
\begin{algorithmic}[1]
			\Require  $\left( {\mathcal{V}, \mathcal{A}, \mathcal{W}}
			\right)$: road network; ${\bx^*}$: the observed flow vector; ${\bg^0}$: the initial demand vector; $\rho$, $T$: two positive integer parameters;
			$\epsilon_1 \geq 0$, $\epsilon_2 > 0$: two real parameters. 
			\State \textbf{Step 0:} Initialization. Using $\bg^{{0}}$ and ${\bx^*}$ as the input, solve \eqref{inverVI2-multi} to obtain $\bbeta^{{0}}$. Plug $\bbeta^{{0}}$ into \eqref{costEstimator} to obtain an initial cost function $\hat f^0(\cdot)$; then, with this cost function and the demand vector $\bg^{{0}}$, solve the forward problem $\text{VI}(\bt,\scrF)$ (using the Method of Successive Averages (MSA) \cite{noriega2007algorithmic}; see Alg.~\ref{alg:msa}) to obtain $\bx^{{0}}$. Set $l=0$. If $F\left(\bbeta^0,\bg^{{0}}\right) = 0$, terminate and return $\bbeta^0$, $\bg^0$; otherwise, go onto Step 1.
			\State \textbf{Step 1:} Computation of a descent direction. Calculate ${\bh^l} =  - \nabla_{\bg} F\left(\bbeta^l, {{\bg ^l}} \right)$ by \eqref{gradient}.
			\State \textbf{Step 2:} Calculation of a search direction. For $i \in [\kern-0.15em[ \scrW 
			]\kern-0.15em], \,u \in [\kern-0.15em[ \scrU 
			]\kern-0.15em]$, set 
			$$\bar h_{iu}^l = \left\{ \begin{gathered}
			h_{iu}^l,{\text{  if }}\left( {g _{iu}^l > \epsilon_1 } \right){\text{ or }}\left( {g _{iu}^l \leq \epsilon_1 {\text{ and }}h_{iu}^l > 0} \right){\text{, }} \hfill \\
			0,{\text{  otherwise.}} \hfill \\ 
			\end{gathered}  \right.$$
			\State \textbf{Step 3:} Determination of a step-size using Armijo-type line search. 
			\begin{itemize}
				\item[] 		\textbf{3.1:} Calculate the maximum possible step-size $\alpha _{\max }^l = \min \left\{ { - {{g _{iu}^l}}\big/{{\bar h_{iu}^l}}; \, \bar h_{iu}^l < 0, \,i \in [\kern-0.15em[ \scrW 
					]\kern-0.15em], \,u \in [\kern-0.15em[ \scrU 
					]\kern-0.15em]} \right\}$. 
				\item[] 		\textbf{3.2:} Determine ${\alpha ^l} = \mathop {\arg \min }\limits_{\alpha  \in \scrS} F\left(\bbeta^l, {{\bg ^l} + \alpha {{\bar {\bh}}^l}} \right)$ with $\scrS \defeq \left\{{\alpha _{\max }^l,{{\alpha _{\max }^l} \mathord{\left/
							{\vphantom {{\alpha _{\max }^l} \rho }} \right.
							\kern-\nulldelimiterspace} \rho },{{\alpha _{\max }^l} \mathord{\left/
							{\vphantom {{\alpha _{\max }^l} {{\rho ^2}}}} \right.
							\kern-\nulldelimiterspace} {{\rho ^2}}}, \ldots ,{{\alpha _{\max }^l} \mathord{\left/
							{\vphantom {{\alpha _{\max }^l} {{\rho ^{T - 1}}}}} \right.
							\kern-\nulldelimiterspace} {{\rho ^{T}}}}}, 0 \right\}$.
			\end{itemize}
			\State \textbf{Step 4:} Update and termination. 
			\begin{itemize}
				\item[] \textbf{4.1:} 
					Set ${\bg ^{l + 1}} = {\bg ^l} + {\alpha ^l}{{\bar \bh}^l}$. If $\frac{{F\left(\bbeta^l, {{\bg^l}} \right) - F\left(\bbeta^{l}, {{\bg^{l + 1}}} \right)}}{{F\left(\bbeta^0, {{\bg^0}} \right)}} < {\epsilon _2}$, stop iteration and return $\bbeta^{l}$, $\bg^{l+1}$; otherwise, go onto Step 4.2.
				\item[] \textbf{4.2:} Using $\bg^{{l+1}}$ and ${\bx^*}$ as the input, solve \eqref{inverVI2-multi} to obtain $\bbeta^{{l+1}}$. Plug $\bbeta^{{l+1}}$ into \eqref{costEstimator} to obtain a new cost function $\hat f^{l+1}(\cdot)$; then, with this cost function and the demand vector $\bg^{{l+1}}$, solve $\text{VI}(\bt,\scrF)$ to obtain $\bx^{{l+1}}$. If $F(\bbeta^{l+1},\bg^{l+1}) > F(\bbeta^l,\bg^{l+1})$, reset $\bbeta^{l+1} = \bbeta^l$. Set $l=l+1$ and return to Step 1.	
			\end{itemize}
	\end{algorithmic}
\end{algorithm}

	
\begin{prop} \label{prop1} Alg.~\ref{alg:joint} converges. \end{prop}
\pf  If the initial demand vector $\bg^{{0}}$ satisfies
    $F\left(\bbeta^0,\bg^{{0}}\right) = 0$, then, by Step 0, the
    algorithm terminates (trivial case). Otherwise, we have
    $F\left(\bbeta^0,\bg^{{0}}\right) > 0$, and it is seen from
    \eqref{bilevel} that the objective function
    $F\left(\bbeta,\bg\right)$ has a lower bound $0$. In addition, by
    the line search and the update steps (Steps 3.2 and 4.1, in
    particular), we obtain
\begin{align} 
F\left(\bbeta^{l}, {{\bg^{l + 1}}} \right) = F\left(\bbeta^l, {{\bg^l} + {\alpha ^l}{{\bar \bh}^l}} \right) = \mathop {\min }\limits_{\alpha  \in \scrS} F\left(\bbeta^l, {{\bg^l} + \alpha {{\bar \bh}^l}} \right) \leq F\left(\bbeta^l, {{\bg^l}} \right), \, \forall l, \label{nonInc}
\end{align}
where the last inequality holds due to $0 \in \scrS$. In addition, by
Step 4.2 it is assured that $F\left(\bbeta^{l+1}, {{\bg^{l + 1}}}
\right) \le F\left(\bbeta^{l}, {{\bg^{l + 1}}} \right) $ (cf. Remark
\ref{rmk:alg1}), which, combined with \eqref{nonInc}, indicates that the
nonnegative objective function $F(\bbeta,\bg)$ in \eqref{bilevel} is
non-increasing. Thus, by the well-known monotone convergence theorem,
the convergence of the algorithm can be guaranteed.
\qed
	
	
\begin{rmk} \label{rmk:alg1} \em{Regarding Steps 0 and 4.2 of
    Alg.~\ref{alg:joint}, a natural question could be raised: Shouldn't
    $\bx^{l}$ ($\forall l \ge 0$) obtained in that way be equal to
    $\bx^*$? We note that $\bx^{l}$ is typically not equal to ${\bx^*}$,
    because ${\bx^*}$ does not necessarily satisfy flow conservations
    for all OD pairs corresponding to $\bg^{l}$ when $\bg^{l} \ne \bg^*$
    ($\bg^*$ denotes the ground-truth OD demand vector), but $\bx^{l}$
    does. To put it another way, if $\bg^{l} \ne \bg^*$, then we have
    $\bx^* \notin \scrF^l$, but $\bx^l \in \scrF^l$, where $\scrF^l$ is
    the feasible set $\scrF$ (see \eqref{feasi}) instantiated under
    demand vector $\bg^l$. It is also worth pointing out that,
    mathematically, such schemes for ``tuning'' the cost function as
    elaborated in Steps 0 and 4.2, would serve to help decrease the
    objective function value such that $F(\bbeta^{l+1},\bg^{l+1}) \le
    F(\bbeta^l,\bg^{l+1})$.}
\end{rmk}
	
	
	%%%%%%%%%%%%%%%%%%%%%%%%%%%%%%%%%%%%%%%%%%%%%%%%%%%%%%%%%%%%%%%
	\begin{algorithm}
	\caption{Method of Successive Averages (MSA) \cite{noriega2007algorithmic}}
	\label{alg:msa}
	\begin{algorithmic}[1]
		\Require $\big( {\tilde\scrV, \tilde\scrA, \tilde\scrW} \big)$: road network; $\tilde{\scrU}$: set of vehicle classes; $f(\cdot)$: cornerstone cost function in \eqref{costMulti}; $\bg_u$, $u \in [\kern-0.15em[ \tilde\scrU 
		]\kern-0.15em]$: demand vector for vehicle class $u$; $\epsilon_3 > 0$: a real parameter; $L$: maximum number of iterations. 
		\State \textbf{Step 0:} Initialization. Initialize link flows $x^{\ell}_{iu} = 0$ for $i \in [\kern-0.15em[ \tilde\scrA]\kern-0.15em], u \in [\kern-0.15em[ \tilde\scrU 
		]\kern-0.15em] $; set iteration counter $\ell=0$.
		\State \textbf{Step 1:} Compute new extremal flows. Set $\ell = \ell + 1$.
		
		\begin{itemize}
			\item[] \textbf{1.1:} Update link travel costs based on current link flows: $t^{\ell}_{iu} = {t_{iu}}( {{x^{\ell-1}_{i1}}, \ldots ,{x^{\ell-1}_{i|\tilde\scrU|}}} ), \, \forall i \in [\kern-0.15em[ \tilde\scrA 
			]\kern-0.15em], u \in [\kern-0.15em[ \tilde\scrU 
			]\kern-0.15em]$.
			
			\item[] \textbf{1.2:} Carry out ``all-or-nothing'' assignment of the demands $\bg_u$ on current shortest paths to obtain $y^{\ell}_{iu}$.
			
		\end{itemize}
		\State \textbf{Step 2:} Update link flows via
		\[x^{\ell}_{iu} = x^{\ell-1}_{iu} + {\lambda ^\ell}\left( {y^{\ell}_{iu} - x^{\ell-1}_{iu}} \right),\]
		where ${\lambda ^\ell} = {1 \mathord{\left/
				{\vphantom {1 l}} \right.
				\kern-\nulldelimiterspace} \ell}$.
		
		\State \textbf{Step 3:} Stopping criterion (slightly different than that in \cite{noriega2007algorithmic}).
		Compute the \textit{Relative Gap} ($\textit{RG}$)
		\[\text{RG} = \frac{{{{\left\| {{\bx^{\ell}} - {\bx^{\ell-1}}} \right\|}}}}{{{{\left\| {{\bx^{\ell}}} \right\|}}}}.\]
		If $\text{RG} < \epsilon_3$ or $\ell \ge L$, terminate; otherwise, return to Step 1.
	\end{algorithmic}
\end{algorithm}

	
	
\begin{rmk}
	\em{Alg.~\ref{alg:joint} is a variant of the algorithms proposed in \cite{spiess1990gradient} and \cite{lundgren2008heuristic}. We use a different method to calculate the \textit{step-sizes} (resp., \textit{Jacobian matrix}) than that in \cite{spiess1990gradient} (resp., \cite{lundgren2008heuristic}). 
	Moreover, as a subroutine, Alg.~\ref{alg:msa} is borrowed from \cite{noriega2007algorithmic}, which has the advantages of being easy to implement and numerically stable when applied to multi-class models.
	}
\end{rmk}

\begin{rmk} \label{rem:gamma1} \em{Let us fix $\gamma_2=1$ in
    \eqref{bilevel} hereafter. Intuitively, the closer the initial
    $\bg^0$ to the ground truth $\bg^*$, the larger the $\gamma_1$ we
    should set; otherwise the contribution of the first term to the
    objective function will be small. In practice, however, we typically
    do not have exact information about how far $\bg^0$ is from $\bg^*$;
    we therefore need to appropriately tune $\gamma_1$. One possible
    criterion is that, fixing the parameters involved in
    Alg.~\ref{alg:joint}, a ``good'' $\gamma_1$ should lead to a
    reduction of the objective function value of BiLev as much as
    possible.  In Sec.~\ref{sec:num} we will present our findings of how
    to set an appropriate $\gamma_1$ through numerical studies using a
    benchmark network.}
\end{rmk}

	
\begin{rmk}
  \em{Note that, formally, if $|\tilde \scrU| = 1$, then all the above
    analytical results derived for the multi-class transportation
    networks reduce to the single-class case. We should keep in mind,
    however, that the behavior of the two models could be quite
    different. For instance, Assump. \ref{cdc16-assumption1} (the strong
    monotonicity assumption) on the cost function is hard to verify
    for general multi-class networks ($|\tilde \scrU| > 1$).}
\end{rmk}
	
	
\section{Numerical results} \label{sec:num}


In this section, we elaborate on numerical experiments conducted on both
benchmark networks \cite{BarGera16} and an actual road network. First,
over three benchmark networks ranging from small-sized to large-sized
(Sioux-Falls, Berlin-Tiergarten, and Anaheim) we adopt the single-class
model by assuming only one class of vehicles (private cars). Then, for
each of the benchmark networks, we simulate a multi-class model by
assuming we have both private cars, indexed by Class 1, and commercial
trucks, indexed by Class 2, thus ending up with a multi-class model
where $|\tilde \scrU| = 2$. Finally, leveraging actual traffic data and
recently developed techniques, we apply our approach to a large highway
network of Eastern Massachusetts (EMA), over which both a single-class
(private cars only) model and a multi-class ($|\tilde \scrU| = 2$; both
private cars and commercial trucks) model are considered.

In all the benchmark network single-class model scenarios, for each OD
pair, we obtain the initial demand by scaling the ground truth demand
(available via \cite{BarGera16}) with a uniform distribution over $[0.9,
1.1]$.  In all the benchmark network multi-class model scenarios, we
divide the original ground truth demand (available via \cite{BarGera16})
proportionally, $80\%$ for private cars and $20\%$ for commercial
trucks, to obtain the ground truth demand vectors accordingly. For each
OD pair and for each vehicle class, the initial demand is obtained by
scaling the corresponding ground truth demand with a uniform
distribution over $[0.9, 1.1]$. On the other hand, for the EMA highway
network scenarios, we leverage real traffic data to estimate the initial
demand vectors; for details, see the Appendix.


For both single-class and multi-class models, we take the ground truth
$f\left(\cdot\right)$ to be $f\left(z\right) = 1 + 0.15z^4, \;z \geq 0$,
for the Sioux-Falls and Anaheim networks, and $f\left(z\right) = 1 +
z^4, \;z \geq 0$, for the Berlin-Tiergarten network. Whereas, for the
EMA highway subnetwork we do not have a ground truth
$f\left(\cdot\right)$; we take $f\left(z\right) = 1 + 0.15z^4, \;z \geq
0$, (the well-known BPR function) as a reference. Regardless of the
benchmark networks or the EMA highway network, in all the multi-class
model scenarios, taking account of the relatively lower speeds and
larger sizes for trucks compared to cars, we assume the flow weight
vector to be $\btheta = (1.0, 2.0)$, and assume ${t^{{0}}_{i1}} = 1.0
\times t^{{0}}_{i}$ and ${t^{{0}}_{i2}} = 1.1 \times t^{{0}}_{i}$, where
$t^{{0}}_{i}$ is the \textit{reference free-flow travel time} (available
via \cite{BarGera16}) for any \textit{physical} link indexed by $i$.

For each of the benchmark network, the observed flow vector ${\bx^*}$ is
generated by solving the forward problem $\text{VI}(\bt,\scrF)$ via
Alg.~\ref{alg:msa} (with the corresponding ground truth demand and cost
function as input). While for the EMA highway network, ${\bx^*}$ is
inferred from the collected actual speed data and the flow capacity
data; see the Appendix for details.

First, to investigate how the settings of $\gamma_1$ and $\gamma_2$ in
\eqref{bilevel} would affect the performance of Alg.~\ref{alg:joint} (in
terms of the reduction of the objective function value of BiLev), we
conduct numerical experiments over the Sioux-Falls benchmark network
single-class model by fixing $\gamma_2 = 1$ and letting $\gamma_1$ vary
over a candidate set $\{0, 0.001, 0.01, 0.1, 1, 10, 100, 1000\}$.  We
set $n=6$, $c = 3.5$, and $\gamma = 1.0$ in
\eqref{inverVI2-multi}. When implementing Algs.~\ref{alg:joint} and
\ref{alg:msa}, we set $\rho = 2$, $T = 10$, $\epsilon_1 =0$, $\epsilon_2
= 10^{-20}$, $\epsilon_3 = 10^{-6}$, and $L = 1000$. The key outputs are
summarized in Tab. \ref{tab3}, from which we see that under the
particular settings of this study, a relatively small $\gamma_1$ ($0 \le
\gamma_1 \le 1$) would lead to a significant reduction of the objective
function value of BiLev. We should keep in mind, however, that as noted
in Remark \ref{rem:gamma1}, tuning $\gamma_1$ would be typically
inevitable for a different scenario, and we could select a ``good''
$\gamma_1$ that would lead to a reduction of the objective function
value of BiLev as much as possible.


Next, to ensure comparison fairness and without loss of generality, for
all scenarios we universally take $\gamma_1 = 1$ and $\gamma_2=1$ in
\eqref{bilevel}. When implementing Algs.~\ref{alg:joint} and
\ref{alg:msa}, we set $\rho = 2$, $T = 10$, $\epsilon_1 =0$, $\epsilon_2
= 10^{-20}$, $\epsilon_3 = 10^{-6}$, and $L = 1000$; these parameters
could be tuned representing the trade-off between the computation burden
and the output accuracy.

In the following, for each and every benchmark network we plot three
types of key quantities at each iteration $l$ of Alg.~\ref{alg:joint}:
the normalized objective function value $F^l/F^0 \defeq
F(\bbeta^l,\bg^l)/F(\bbeta^0,\bg^0)$, the normalized demand difference
w.r.t. ground truth $\|\bg^l-\bg^*\|/\|\bg^*\|$ for the single-class
model (resp., $\|\bg_u^l-\bg_u^*\|/\|\bg_u^*\|$ for the multi-class
model), where $ \bg^*$ (resp., $\bg_u^*$) denotes the ground truth, and
the cost function estimates $\hat f^l(\cdot)$. On the other hand, for
the EMA highway network, since we do not have the ground truth demand
vector, we only plot the other two types of key quantities. Moreover,
for convenience of comparisons, in Tabs. \ref{tab1} and \ref{tab2} we
summarize some of the results in detail.

\subsection{Sioux-Falls network}

The Sioux-Falls network contains 24 zones (hence, $24 \times (24 - 1) =
552$ OD pairs), 24 nodes, and 76 links.

\subsubsection*{Single-class model} 
We use $n=6$, $c = 3.5$, and $\gamma = 1.0$ in \eqref{inverVI2-multi};
note that, for a given network, using simulated data (demand vectors and
the corresponding Wardrop equilibrium link flows), the values of $n,
\;c, \;\gamma$ can be determined by cross-validation and then applied to
new data. We plot selected outputs after executing Alg.~\ref{alg:joint}
in Fig.~\ref{fig:uni_Sioux}. Fig.~\ref{fig:uni_objFun_Sioux} shows that,
after 20 iterations, the objective function value of BiLev is reduced by
more than $80\%$. Fig.~\ref{fig:uni_demandsDiff_biLev_Sioux} shows that,
although the distance between the adjusted demand and the ground truth
demand does not keep decreasing as the iteration count increases, the
distance changes very slightly, meaning that the adjustment procedure
does not alter the initial demand much. Fig.~\ref{fig:uni_CostSioux}
shows that, a ``rough'' initial demand vector, combined with the
observed flow vector, can already enable us to obtain a very accurate
estimate for the cost function and, as the iteration count increases,
the estimate of the cost function gets closer and closer to the ground
truth (see the green curve corresponding to $l=20$, which is the closest
to the red ground truth).

\subsubsection*{Multi-class model} 
We use $n=6$, $c = 3.5$, and $\gamma = 1.0$ in \eqref{inverVI2-multi},
and we plot the outputs in Fig.~\ref{fig:Sioux}. Fig.~\ref{objFun_Sioux}
shows that, after 5 iterations, the objective function value of BiLev is
reduced by about $35\%$. Fig.~\ref{demandsDiff_biLev_Sioux} shows that,
for either vehicle class (private car or commercial truck), though the
distance between the adjusted demand and the ground truth would not
always keep decreasing as the iteration count increases, the distance
changes very slightly, meaning that the adjustment procedure does not
alter the initial demand much. Fig.~\ref{fig:CostSioux} shows that the
initial estimate of the cost function is already very close to the
ground truth and, as iterations progress, the estimates remain very
close to the ground truth.


\begin{figure}[H]  
	\centering
	\begin{subfigure}[b]{0.8\textwidth}
		\includegraphics[width=\textwidth]{imag/objFun_Sioux_1_1.pdf}
		\caption{}
		\label{fig:uni_objFun_Sioux}
	\end{subfigure} 
	\begin{subfigure}[b]{0.8\textwidth}
		\includegraphics[width=\textwidth]{imag/demandsDiff_biLev_Sioux_1_1.pdf}
		\caption{}
		\label{fig:uni_demandsDiff_biLev_Sioux}
	\end{subfigure}   
	\begin{subfigure}[b]{0.8\textwidth}
		\includegraphics[width=\textwidth]{imag/fitting_Sioux_1_1.pdf}
		\caption{}
		\label{fig:uni_CostSioux}
	\end{subfigure}  
	\caption{Key quantities and cost function estimations w.r.t.\
          the joint iterations (Sioux-Falls; single-class): (a)
          normalized objective function values; (b) demand differences
          w.r.t.\ ground truth; (c) cost function estimates.}
	\label{fig:uni_Sioux}
\end{figure}
	
\begin{figure}[H]  
	\centering
	\begin{subfigure}[b]{0.8\textwidth}
		\includegraphics[width=\textwidth]{imag/objFun_Sioux_multi_class_1_1.pdf}
		\caption{}
		\label{objFun_Sioux}
	\end{subfigure} 
	\begin{subfigure}[b]{0.8\textwidth}
		\includegraphics[width=\textwidth]{imag/demandsDiff_biLev_Sioux_multi_class_1_1.pdf}
		\caption{}
		\label{demandsDiff_biLev_Sioux}
	\end{subfigure}   
	\begin{subfigure}[b]{0.8\textwidth}
		\includegraphics[width=\textwidth]{imag/fitting_Sioux_multi_class_1_1.pdf}
		\caption{}
		\label{fig:CostSioux}
	\end{subfigure}  
	\caption{Key quantities and cost function estimations w.r.t.\
          the joint iterations (Sioux-Falls; multi-class): (a)
          normalized objective function values; (b) demand differences
          w.r.t.\ ground truth; (c) cost function estimates.}
	\label{fig:Sioux}
\end{figure}
	
	
\subsection{Berlin-Tiergarten network}

The Berlin-Tiergarten network \cite{BarGera16} contains 26 zones (hence,
$26 \times (26 - 1) = 650$ OD pairs), 361 nodes, and 766 links.

\subsubsection*{Single-class model}
 We use $n=6$, $c = 0.5$, and $\gamma = 0.001$ in \eqref{inverVI2-multi}.
See Fig.~\ref{fig:uni_Tiergarten} for the
results. Fig.~\ref{fig:uni_objFun_Tiergarten} shows that, after 19
iterations, the objective function value of BiLev is reduced by more
than $50\%$. Similar to the Sioux-Falls network single-class scenario,
from Fig.~\ref{fig:uni_demandsDiff_biLev_Tiergarten} we see that the
distance between the adjusted demand and the ground truth does not keep
decreasing as iterations progress, but still, the distance changes very
slightly, suggesting that the adjustment procedure does not alter the
initial demand significantly. It is also seen from
Fig.~\ref{fig:uni_CostTiergarten} that, as iteration count increases,
the estimate of the cost function improves to some degree.  


\subsubsection*{Multi-class model}
We use $n=7$, $c = 1.5$, and $\gamma = 0.1$ in \eqref{inverVI2-multi}.
See Fig.~\ref{fig:Tiergarten} for the
results. Fig.~\ref{objFun_Tiergarten} shows that, after 9 iterations,
the objective function value of BiLev is reduced by more than
$12\%$. Fig.~\ref{demandsDiff_biLev_Tiergarten} shows that, for either
vehicle class (private car or commercial truck), the distance between
the adjusted demand and the ground truth keeps decreasing and the
distance changes very slightly, meaning the adjustment procedure does
not alter the initial demand much. Similar to the Sioux-Falls network
multi-class scenario, from Fig.~\ref{fig:CostTiergarten} we see that,
the initial estimate for the cost function is already very close to the
ground truth, and as iterations advance, the estimates remain very close
to the ground truth. 

	
\begin{figure}[H]  
	\centering
	\begin{subfigure}[b]{0.8\textwidth}
		\includegraphics[width=\textwidth]{imag/objFun_Tiergarten_1_1.pdf}
		\caption{}
		\label{fig:uni_objFun_Tiergarten}
	\end{subfigure} 
	\begin{subfigure}[b]{0.8\textwidth}
		\includegraphics[width=\textwidth]{imag/demandsDiff_biLev_Tiergarten_1_1.pdf}
		\caption{}
		\label{fig:uni_demandsDiff_biLev_Tiergarten}
	\end{subfigure}   
	\begin{subfigure}[b]{0.8\textwidth}
		\includegraphics[width=\textwidth]{imag/fitting_Tiergarten_1_1.pdf}
		\caption{}
		\label{fig:uni_CostTiergarten}
	\end{subfigure}   
	\caption{Key quantities and cost function estimations w.r.t.\
          the joint iterations (Berlin-Tiergarten; single-class): (a)
          normalized objective function values; (b) demand differences
          w.r.t.\ ground truth; (c) cost function estimates.}
	\label{fig:uni_Tiergarten}
\end{figure}

\begin{figure}[H]  
	\centering
	\begin{subfigure}[b]{0.8\textwidth}
		\includegraphics[width=\textwidth]{imag/objFun_Tiergarten_multi_class_1_1.pdf}
		\caption{}
		\label{objFun_Tiergarten}
	\end{subfigure} 
	\begin{subfigure}[b]{0.8\textwidth}
		\includegraphics[width=\textwidth]{imag/demandsDiff_biLev_Tiergarten_multi_class_1_1.pdf}
		\caption{}
		\label{demandsDiff_biLev_Tiergarten}
	\end{subfigure}   
	\begin{subfigure}[b]{0.8\textwidth}
		\includegraphics[width=\textwidth]{imag/fitting_Tiergarten_multi_class_1_1.pdf}
		\caption{}
		\label{fig:CostTiergarten}
	\end{subfigure}   
	\caption{Key quantities and cost function estimations w.r.t.\
          the joint iterations (Berlin-Tiergarten; multi-class): (a)
          normalized objective function values; (b) demand differences
          w.r.t.\ ground truth; (c) cost function estimates.}
	\label{fig:Tiergarten}
\end{figure}

	
\subsection{Anaheim network}

The Anaheim network \cite{BarGera16} contains 38 zones (hence, $38
\times (38 - 1) = 1406$ OD pairs), 416 nodes, and 914 links.

\subsubsection*{Single-class model}
We use $n=6$, $c = 3.5$, and $\gamma = 1.0$ in
\eqref{inverVI2-multi}. See Fig.~\ref{fig:uni_Anaheim} for the
results. Similar observations can be made as those for the Sioux-Falls
network single-class scenario; a minor difference is that (see
Fig.~\ref{fig:uni_demandsDiff_biLev_Anaheim}) the distance between the
adjusted demand and the ground truth demand keeps decreasing as the
iteration count increases.
	
\subsubsection*{Multi-class model}
We use $n=6$, $c = 1.5$, and $\gamma = 0.1$ in
\eqref{inverVI2-multi}. See Fig.~\ref{fig:Anaheim} for the
results. Similar observations can be made as those for the
Berlin-Tiergarten network multi-class scenario; a slight difference is
that (see Fig.~\ref{fig:CostAnaheim}), the initial estimate for the cost
function is a bit far way from the ground truth, and this prevents the
iterative cost function estimates produced by our gradient-based method
from getting very close to the ground truth.

\begin{figure}[H]  
	\centering
	\begin{subfigure}[b]{0.8\textwidth}
		\includegraphics[width=\textwidth]{imag/objFun_Anaheim_1_1.pdf}
		\caption{}
		\label{fig:uni_objFun_Anaheim}
	\end{subfigure} 
	\begin{subfigure}[b]{0.8\textwidth}
		\includegraphics[width=\textwidth]{imag/demandsDiff_biLev_Anaheim_1_1.pdf}
		\caption{}
		\label{fig:uni_demandsDiff_biLev_Anaheim}
	\end{subfigure}   
	\begin{subfigure}[b]{0.8\textwidth}
		\includegraphics[width=\textwidth]{imag/fitting_Anaheim_1_1.pdf}
		\caption{}
		\label{fig:uni_CostAnaheim}
	\end{subfigure}  
	\caption{Key quantities and cost function estimations w.r.t.\
          the joint iterations (Anaheim; single-class): (a) normalized
          objective function values; (b) demand differences w.r.t.\
          ground truth; (c) cost function estimates.}
	\label{fig:uni_Anaheim}
\end{figure}
	

\begin{figure}[H]  
	\centering
	\begin{subfigure}[b]{0.8\textwidth}
		\includegraphics[width=\textwidth]{imag/objFun_Anaheim_multi_class_1_1.pdf}
		\caption{}
		\label{objFun_Anaheim}
	\end{subfigure} 
	\begin{subfigure}[b]{0.8\textwidth}
		\includegraphics[width=\textwidth]{imag/demandsDiff_biLev_Anaheim_multi_class_1_1.pdf}
		\caption{}
		\label{demandsDiff_biLev_Anaheim}
	\end{subfigure}   
	\begin{subfigure}[b]{0.8\textwidth}
		\includegraphics[width=\textwidth]{imag/fitting_Anaheim_multi_class_1_1.pdf}
		\caption{}
		\label{fig:CostAnaheim}
	\end{subfigure}  
	\caption{Key quantities and cost function estimations w.r.t.\
          the joint iterations (Anaheim; multi-class): (a) normalized
          objective function values; (b) demand differences w.r.t.\
          ground truth; (c) cost function estimates.}
	\label{fig:Anaheim}
\end{figure}


\begin{table*}[hbt] 
	\centering
	\caption{Summary of Sioux-Falls network single-class scenario
          experiments:  $({\gamma_1},\gamma_2)$ settings and key outputs
          (fix ground-truth $f(z) = 1 + 0.15z^4$ and set $(n,c,\gamma) =
          (6,3.5,1.0)$).} \label{tab3} 
	\resizebox{\linewidth}{!}
	{\begin{tabular}{lll|lll}
            \toprule[1.2pt]
            \textbf{$({\gamma_1},\gamma_2)$ setting}  & \textbf{iterations} & \textbf{$F(\bbeta,\bg)$ reduction} &  \textbf{$({\gamma_1},\gamma_2)$ setting}  & \textbf{iterations} & \textbf{$F(\bbeta,\bg)$ reduction}  \\
            \midrule[1.2pt]
            $(\textbf{0},1)$  &  20  & 96.36\% &$(\textbf{1},1)$  &  20  & 81.94\% \\  
            \midrule[1.2pt]
            $(\textbf{0.001},1)$  &  24  & 98.12\% & $(\textbf{10},1)$  &  11  & 45.72\% \\ 
            \midrule[1.2pt]
            $(\textbf{0.01},1)$ &   19  & 89.42\% & $(\textbf{100},1)$  &  3  & 13.15\% \\ 
            \midrule[1.2pt]
            $(\textbf{0.1},1)$  &  42  & 97.56\%  & $(\textbf{1000},1)$ &  2  & 2.83\% \\ 
            \bottomrule[1.2pt]
		\end{tabular}}
	\end{table*}
		
	
	
\begin{table*}[hbt] 
	\centering
	\caption{Summary of benchmark network experiments: sizes,
          parameter settings, and key outputs.} \label{tab1} 
	\resizebox{\linewidth}{!}
	{\begin{tabular}{lllllll}
			\toprule[1.2pt]
			\textbf{network}  & \textbf{zones/nodes/links} & \textbf{ground-truth $f(\cdot)$} & \textbf{model type} & \textbf{$(n,c,\gamma)$ setting}  & \textbf{iterations} & \textbf{$F(\bbeta,\bg)$ reduction}  \\
			\midrule[1.2pt]
			\multirow{2}{*}{Sioux-Falls} & \multirow{2}{*}{24/24/76} & \multirow{2}{*}{$f(z) = 1 + 0.15z^4$} & single-class & $(6,3.5,1.0)$ &  20  & 81.94\% \\ & & & multi-class  & $(6,3.5,1.0)$ &  5  & 36.63\% \\
			\midrule[1.2pt]
			\multirow{2}{*}{Tiergarten}  & \multirow{2}{*}{26/361/766}  & \multirow{2}{*}{$f(z) = 1 + z^4$} & single-class & $(6,0.5,0.001)$  & 19  & 54.42\% \\& & & multi-class & $(7,1.5,0.1)$ &  9  & 12.89\% \\
			\midrule[1.2pt]
			\multirow{2}{*}{Anaheim}  & \multirow{2}{*}{38/416/914}  & \multirow{2}{*}{$f(z) = 1 + 0.15z^4$} & single-class & $(6,3.5,1.0)$ & 10 & 59.33\% \\& & & multi-class & $(6,1.5,0.1)$ &  3  & 45.22\% \\
			\bottomrule[1.2pt]
	\end{tabular}}
\end{table*}

	
\begin{figure}[H]
	\centering
	\includegraphics[width=0.6\textwidth]{imag/unfiltered.png}
	\caption{All available road segments in Eastern Massachusetts (from \cite{CDC16}).}
	\label{eastMA}
\end{figure}

\begin{figure}[htp]  
	\centering
	\includegraphics[width=0.9\textwidth]{imag/ma_journal_map}
	\caption{Eastern Massachusetts highway network; see \cite{InverseVIsTraffic} for the correspondences between nodes and link indices. (``nodes:zone'' pairs -- \{1\}: Seabrook (NH); \{2, 4, 5\}: NH; \{3\}: Haverhill; \{6, 8\}: Lawrence; \{7, 9\}: Georgetown; \{10, 11\}: Lowell; \{12, 15\}: Salem; \{13, 14\}: Peabody; \{16, 17, 18, 19\}: Burlington; \{20\}: Littleton; \{21\}: Lexington; \{22\}: Boston; \{23, 24, 25, 26, 27, 28\}: Waltham; \{29\}: Quincy; \{30, 31, 32, 33, 34\}: Marlborough/Framingham; \{35, 71\}: Milford; \{36\}: Franklin; \{37, 38, 39, 40, 41\}: Westwood/Quincy; \{42\}: Dedham; \{43, 44, 45, 46, 47\}: Foxborough; \{48, 74\}: Taunton; \{49, 73\}: Plymouth; \{50, 51\}: Cape Cod; \{52\}: Dartmouth; \{53\}: Fall River; \{54, 68, 70\}: RI; \{55, 56\}: VT; \{57\}: Westminster; \{58\}: Leominster; \{59, 60, 72\}: Worcester; \{61\}: Amherst; \{62, 63, 64, 65, 66\}: CT; \{67\}: Webster; \{69\}: Uxbridge.)}
	\label{fig:zoneMA}
\end{figure}


\subsection{Eastern Massachusetts highway network}

The Eastern Massachusetts (EMA) highway network contains 34 zones (hence, $34 \times (34 - 1) = 1122$ OD pairs), 74 nodes, and 258 links. We describe the EMA datasets and elaborate the data preprocessing procedures in the Appendix.

\subsubsection*{Single-class model}
Take $n=8$, $c = 1.5$, and $\gamma = 0.001$ in
\eqref{inverVI2-multi}. See Fig.~\ref{fig:ema} for the
results. Fig.~\ref{objFun_ema} shows that, after 3 iterations, the
objective function value of BiLev is reduced by more than $7\%$.
	
\subsubsection*{Multi-class model}
Take $n=8$, $c = 1.5$, and $\gamma = 0.001$ in
\eqref{inverVI2-multi}. See Fig.~\ref{fig:ema-multi} for the
results. Fig.~\ref{objFun-ema-multi} shows that, after 7 iterations, the
objective function value of BiLev is reduced by more than $5\%$.

	
	\begin{figure}[H]  
	\centering
	\begin{subfigure}[b]{0.8\textwidth}
		\includegraphics[width=\textwidth]{imag/objFun_MA_journal.pdf}
		\caption{}
		\label{objFun_ema}
	\end{subfigure} 
	\begin{subfigure}[b]{0.8\textwidth}
		\includegraphics[width=\textwidth]{imag/fitting_MA_journal.pdf}
		\caption{}
		\label{fig:costEma}
	\end{subfigure}  
	\caption{Key quantities and cost function estimations w.r.t.\
          the joint iterations (EMA; single-class): (a) Normalized
          objective function values; (b) Cost function estimation.}
	\label{fig:ema}
\end{figure}
	
	
	\begin{figure}[H]  
	\centering
	\begin{subfigure}[b]{0.8\textwidth}
		\includegraphics[width=\textwidth]{imag/objFun_MA_journal_multi_class_1_1.pdf}
		\caption{}
		\label{objFun-ema-multi}
	\end{subfigure} 
	\begin{subfigure}[b]{0.8\textwidth}
		\includegraphics[width=\textwidth]{imag/fitting_MA_journal_multi_class_1_1.pdf}
		\caption{}
		\label{fig:costEma-multi}
	\end{subfigure}  
	\caption{Key quantities and cost function estimations w.r.t.\
          the joint iterations (EMA; multi-class): (a) Normalized
          objective function values; (b) Cost function estimation.}
	\label{fig:ema-multi}
\end{figure}
	

\begin{table*}[hbt]
	\centering
	\caption{Summary of EMA highway subnetwork experiments: size, parameter settings, and key outputs.} \label{tab2}
	\resizebox{\linewidth}{!}
	{\begin{tabular}{lllllll}
            \toprule[1.2pt]
            \textbf{network}  & \textbf{zones/nodes/links} & \textbf{reference $f(\cdot)$} & model type & \textbf{$(n,c,\gamma)$ setting} & \textbf{iterations} & \textbf{$F(\bbeta,\bg)$ reduction}  \\
            \midrule[1.2pt]
            \multirow{2}{*}{EMA}  & \multirow{2}{*}{34/74/258} & \multirow{2}{*}{$f(z) = 1 + 0.15z^4$} & single-class & $(8,1.5,0.001)$ & 3 & {7.38\%} \\& & & multi-class & $(8,1.5,0.001)$ & 7 & 5.11\% \\
            \bottomrule[1.2pt]
	\end{tabular}}
\end{table*}

	
\section{Conclusions and future work}  \label{sec:conc}
	
	
From the numerical results for benchmark networks we see that
Alg.~\ref{alg:joint} (with Alg.~\ref{alg:msa} as a subroutine) works
well in terms of reducing the objective function value of BiLev while
improving the estimation accuracy for the cost function. Though we can
not always expect that the two goals can be achieved simultaneously, our
experience shows that we can always reduce the objective function value
of BiLev to some extent, sometimes significantly; this is due to the
construction of Alg.~\ref{alg:joint} (see Prop. \ref{prop1}). In
addition, we have not observed cases where the cost function estimate
deteriorates as iterations advance. Although we can not always expect
the adjusted OD demands to be closer to the ground truth compared to
their initial values, we see that in all our experiments, the actual
adjustments on the initial demand are slight, meaning we do not alter
the original data much. Furthermore, through the numerical experiments
over the EMA highway network, we demonstrate how our proposed algorithms
can be applied using real data.
	
Note that, of course, in practice, the output of our algorithm would
heavily depend on the initial demand data as well as the accuracy of the
flow observations. It is also worth mentioning that the two subroutines
for solving the forward VI problem (Alg.~\ref{alg:msa}) and the inverse
VI problem (we solve \eqref{inverVI2-multi} using the Gurobi solver)
would largely determine the total running time. Our experience suggests
that more accurate outputs of these two subroutines would typically
increase the number of iterations in Alg.~\ref{alg:joint}. Take the
subroutine Alg.~\ref{alg:msa} for instance; in practice one may want to
save some computation time by setting the parameter $L$ (resp.,
$\epsilon_3$) to a smaller (resp., bigger) value, thus sacrificing a
little accuracy of the final OD demand vector and cost function.
	
Possible future directions include:
\begin{itemize}
\item[-] As indicated in our earlier work \cite{ieee18}, a potential
  more accurate estimate of the cost functions could be leveraged to
  facilitate smarter GPS navigation, thus reducing congestion (or, more
  specifically the ``Price of Anarchy''~\cite{CDC16,IFAC17,ieee18}) in
  road networks. It is also possible to perform sensitivity analysis of
  link congestion metrics with respect to key quantities, such as link
  capacity and free-flow speed, similar to the analysis we have
  conducted in \cite{ieee18} for single-class networks. The results of
  this analysis would enable the Transportation department of a city to
  prioritize congestion-reducing interventions.
 
\item[-] One can consider integrating our algorithms to a dynamic OD
  demand estimation problem setting; see, e.g.,
  \cite{pitombeira2017dynamic}, among others. The potential outcome
  would be to provide a more trustworthy database so as to predict the
  ``Estimated Time of Arrival'' \cite{eta} more accurately.
\end{itemize}

	
\section*{Appendix}   \label{sec:append}

We first provide a brief description of the datasets extracted from the
Eastern Massachusetts (EMA) road network, and then elaborate on the data
preprocessing.

\subsection*{\Rmnum{1}. Description of datasets} \label{sec: dataset}

Thanks to the Boston Region Metropolitan Planning Organization (MPO), we have access to two datasets over the EMA road network: 
\noindent (\rmnum{1}) The speed dataset includes the spatial average
speeds (indistinguishable among different vehicle classes) for more than
13,000 road segments (with an average length of 0.7 miles; see
Fig.~\ref{eastMA}) of EMA, providing the average speed for every minute
of the year 2012.  For each road segment, identified with a unique {\em
  tmc (traffic message channel)} code, the dataset provides information
such as speed data (instantaneous, average, and free-flow speed) in
\emph{miles per hour (mph)}, date and time, and traveling time (in
\emph{minute}) through that segment.
\noindent (\rmnum{2}) The flow capacity (in \emph{vehicles per hour})
dataset includes capacity data for more than 100,000 road segments (with
an average length of 0.13 miles) in EMA. For more detailed information
of the two datasets, see \cite{CDC16}.


\subsection*{\Rmnum{2}. Preprocessing} \label{prep}

\begin{itemize}

	
	\item[\rmnum{1})] \textit{Selecting a sub-network} \label{sec:sele}
	
	To reduce the computational burden while capturing the key elements of the
	Eastern Massachusetts road network, we only consider a representative
	highway sub-network as shown in Fig.~\ref{fig:zoneMA}, where there
	are 2984 road segments, composing a road network $\big( {\tilde {\scrV},\tilde {\scrA}}, \tilde {\scrW} \big)$ with 74 nodes and 258 links. 
	Further, we simplify the analysis by grouping nodes within the same area, assigning them the same \emph{zone} label, thus obtaining 34 zones (as opposed to 74 nodes). Assuming that each zone could be an origin and a destination, then there are $34 \times (34-1) = 1122$ OD pairs. It is worth mentioning that the nodes 72, 73, and 74 are introduced for ensuring the identifiability of the OD demand matrices; see \cite[Lemma 2]{hazelton2000estimation}. We clarify here that, in our recent released EMA benchmark network \cite{BarGera16}, for simplicity we label each and every of the 74 nodes as a \emph{zone}. However, only 34 of them play an actual role in composing OD pairs; to put it another way, there are $74 \times (74 - 1) - 34 \times (34 - 1) = 4280$ ``fictitious'' OD pairs whose flow demand is zero.
	
	\item[\rmnum{2})] \textit{Calculating average speed and free-flow speed} \label{sec:upda}

First, we select a time instances set $\mathcal{T}$ consisting of each
minute of a PM peak period (5 pm -- 7 pm) for each day of April
2012. Note that the selected PM period is a subinterval of
the PM period in the capacity dataset. Then, we calculate
the \textit{average speed} for each road segment over $\scrT$ (covering 120 minutes).  Finally, for each road segment, we compute a reliable
proxy of the \textit{free-flow speed} by using the 85th-percentile point of the
observed speeds on that segment for all the time instances belonging to
$\mathcal{T}$.
	
	\item[\rmnum{3})]	\textit{Aggregating flows of the segments on each link} \label{sec:aggre}

For $i \in [\kern-0.15em[ \tilde\scrA 
]\kern-0.15em]$, let $\{
{v_i^{{j}}},{t_i^{{j}}},{v^{{0j}}_i},{t^{{0j}}_i},{m_i^{{j}}};
\, j = 1, \ldots, J_i \}$ denote the available observations
($v_i^{{j}}$, $t_i^{{j}}$), and parameters
($v^{{0j}}_i$, $t^{{0j}}_i$,
$m_i^{{j}}$) of the segments composing the $i$th \textit{physical} link, where, for each segment $j$, $v^{{j}}_i$
(resp., $v^{{0j}}_i$) is the \emph{average speed} (resp., \emph{free-flow speed};
in \emph{miles per hour}), $t^{{j}}_i$ (resp.,
$t^{{0j}}_i$) is the \emph{travel time} (resp., \emph{free-flow travel
	time}; in \emph{hours}), and $m_i^{{j}}$ is the \emph{flow capacity}
(in \emph{vehicles per hour}). Then, using Greenshield's model \cite{TSM}, we calculate the \textit{traffic
	flow} (in \emph{vehicles per hour}) on segment $j$ by 
\begin{align}
{\hat x_i^{{j}}} = \frac{{4{m_i^{{j}}}}}{{{v^{{0j}}_{i}}}} {v_i^{{j}}} - \frac{{4{m_i^{{j}}}}}{{(v^{{0j}}_{i})^2}} (v_i^{{j}})^2.
\label{Gr}
\end{align}
In our analysis, we enforce ${v_i^{{j}}} \leq
v_i^{{0j}}$ to make sure that the flow given by \eqref{Gr}
is nonnegative. In particular, if for some time instance
${v_i^{{j}}} > v_i^{{0j}}$ (this rarely
happens), we set ${v_i^{{j}}} =
v_i^{{0j}}$ in \eqref{Gr}, thus leading to a zero flow
estimation for this time instance.
Aggregating over all segments composing link $i$ we compute:
\[
\hat x_i = \frac{{\sum\nolimits_{j = 1}^{J_i}
		{{\hat x_i^{{j}}}{t_i^{{j}}}}
}}{{\sum\nolimits_{j = 1}^{J_i} {{t_i^{{j}}}}
}},\,\,
{t_i^{{0}}} = \sum\nolimits_{j = 1}^{J_i}
{{t^{{0j}}_{i}}},\,\, 
{m_i} = \frac{{\sum\nolimits_{j = 1}^{J_i}
		{{m_i^{{j}}}{t^{{0j}}_{i}}}
}}{{\sum\nolimits_{j = 1}^{J_i} {{t^{{0j}}_{i}}} }},  
\]
where $\hat x_i^j$ is given by \eqref{Gr} and ${t^{{0j}}_{i}} =
v_i^{{j}} t_i^{{j}}/v^{{0j}}_{i}$,
$j = 1, \ldots ,J_i$.
	
	
\item[\rmnum{4})] \textit{Adjusting link flows to satisfy conservation}
	\label{sec:flcon}
	
	For $i \in [\kern-0.15em[ \tilde\scrA 
	]\kern-0.15em]$, let ${\hat x}_i$ denote the original estimate of the
	flow on link $i$ (see the last step), $x_i$ its adjustment, and $\xi_{iu}$ the flow percentage on link $i$ for vehicle class $u \in [\kern-0.15em[ \tilde\scrU 
	]\kern-0.15em]$ (note that $\xi_{iu} \ge 0$ and $\sum_{u=1}^{|\tilde{\scrU}|}\xi_{iu} = 1$). Then, $x_{iu} = \xi_{iu} x_i$ (recall that $x_{iu}$ denotes the flow on link $a(i,u)$; i.e., $x_{iu}$ is the flow on link $i$ for vehicle class $u$). We solve the following \emph{Least
		Squares} problem:
	\begin{align}
	\min_{\bx} \quad & \sum_{i = 1}^{  |\tilde\scrA| 
	} \sum_{u = 1}^{ | \tilde\scrU |} {{{\xi^2_{iu}\left( {{x_i} - {{\hat x}_i}} \right)}^2}}  \label{fl1} \\
	{\text{s.t.}} \quad & 
	\sum_{i\in \scrI(v_j)}\xi_{iu}x_i = \sum_{i\in \scrO(v_j)}\xi_{iu}x_i, &\forall j \in [\kern-0.15em[ \tilde\scrV 
	]\kern-0.15em], \, u \in [\kern-0.15em[ \tilde\scrU 
	]\kern-0.15em],
	\notag \\
	\quad &x_i \geq 0, & \forall i \in [\kern-0.15em[ \tilde\scrA 
	]\kern-0.15em], \notag
	\end{align}
	where the first constraint enforces flow conservation for each node $v_j \in \tilde\scrV $ with $\scrI(v_j)$ (resp., $\scrO(v_j)$) denoting the set of links entering
	(resp., outgoing) to (resp., from) node $v_j$. Note that \eqref{fl1} generalizes its counterpart in \cite{CDC16,IFAC17,ieee18}; the latter only tackles the case where $|\tilde{\scrU}| = 1$.
	For the case where $|\tilde{\scrU}| = 2$, the datasets available to us do not contain exact information of the parameters $\xi_{iu}$; in our experiments, we choose universally $\xi_{i1} = 0.8$ (for private cars) and $\xi_{i2} = 0.2$ (for commercial trucks).
	
	\item[\rmnum{5})] 	\textit{Estimating initial OD demand vectors} \label{sec:ODmat}
	
	For network $( {\tilde{\mathcal{V}}, \tilde{\scrA}, \tilde{\mathcal{W}}} )$, to estimate initial OD demand vectors $\bg_u = ( {{g_{iu}}; \,i \in [\kern-0.15em[ \tilde \scrW 
		]\kern-0.15em]} )$, $u \in [\kern-0.15em[ \tilde \scrU 
	]\kern-0.15em]$, we borrow the Generalized Least Squares (GLS) method proposed in \cite{hazelton2000estimation}, which assumes the network to be uncongested (i.e., for each OD pair the route choice probabilities are independent of traffic flow), and the OD trips (traffic counts) to be Poisson distributed. In addition, we assume that the OD demands for different vehicle classes are independent from one another. Fix vehicle class $u \in [\kern-0.15em[ \scrU 
	]\kern-0.15em]$.
	Denote by $\{ {{\bx_u^{\scriptscriptstyle{(k)}}}};
	\, k \in [\kern-0.15em[ \scrK 
	]\kern-0.15em]\}$ $|\scrK|$ observations of the flow vector for vehicle class $u$. Let $\bS_u = (1/(|\scrK| - 1))\sum\nolimits_{k = 1}^{|\scrK|} {\big(
		{{\bx_u^{\scriptscriptstyle{(k)}}} - \bar {\bx}_u} \big){{\big(
				{{\bx_u^{\scriptscriptstyle{(k)}}} - \bar {\bx}_u} \big)}' }}$ be
	the sample covariance matrix, where $\bar \bx_u  = (1/|\scrK|)\sum\nolimits_{k = 1}^{|\scrK|} {{\bx_u^{\scriptscriptstyle{(k)}}}} $. Let $\bP_u = \left[ {{p_u^{ir}}} \right]$ denote the
	route choice probability matrix, where $p_u^{ir}$ is the probability that a driver between OD pair $i$
	selects route $r$. Let $\bxi_u  = \bP_u'\bg_u$. Then, the GLS method proceeds as follows:
	
	(\rmnum{1}) Identify a set $\scrR_u^i$ of feasible routes for each OD pair $i$, thus obtaining the link-route incidence matrix $\bA_u$. 
	
	(\rmnum{2}) 
	Solve sequentially the following two problems:
	\begin{align}
	\text{(P1)} \quad \mathop {\min }\limits_{\bxi_u  \geq \textbf{0}} \quad \frac{|\scrK|}{2}\bxi_u '\bQ_u\bxi_u  - \bb_u'\bxi_u,   \label{qp2}
	\end{align}
	where $\bQ_u = \bA_u'{\bS_u^{ - 1}}\bA_u$ and $\bb_u =
	\sum\nolimits_{k = 1}^{|\scrK|} {\bA_u'{\bS_u^{ - 1}}{\bx_u^{\scriptscriptstyle{(k)}}}} $, and
	\begin{align}
	\text{(P2)} \quad &\mathop {\min }\limits_{\bP_u \geq \textbf{0}, \,\bg_u \geq \textbf{0} } \quad h\left( {\bP_u,\bg_u } \right) \label{qp3} \\
	\text{s.t.} \quad 
	& p_u^{ir} = 0 \quad \forall (i,r) \in \{(i,r): r \notin \scrR_u^i\}, \notag \\
	&\bP_u'\bg_u  = {\bxi_u ^0}, \notag \\
	&\bP_u\textbf{1} = \textbf{1}, \notag 
	\end{align}
	where $h(\cdot, \cdot)$ can be taken as any smooth scalar-valued function, $\bxi_u^0$ is the optimal solution to (P1), and $\textbf{1}$ denotes the vector with all $1$'s as its entries. Note that (P1) (resp., (P2)) is a typical \textit{Quadratic Program (QP)} (resp., \textit{Quadratically Constrained Program (QCP)}). Letting $(\bP_u^0, \bg_u^0)$ be an optimal solution to (P2), then $\bg_u^0$ is our initial estimate of the demand vector for vehicle class $u$.
	It is worth mentioning that here we are able to deal with multi-class models, whereas in \cite{CDC16,IFAC17,ieee18} only single-class models have been considered. 
	
	As pointed out in \cite{ieee18}, we note that the GLS method would encounter numerical difficulties when the network size is as large as the one we are currently dealing with, because there would be too many decision variables in (P2). Thus, we perform a simplification procedure. In particular, we only consider the shortest route for each OD pair of $( {\tilde{\mathcal{V}}, \tilde{\scrA}, \tilde{\mathcal{W}}} )$, thus, leading to a deterministic route choice matrix $\bP$ and significantly reducing the number of decision variables in the QCP (P2).
	
	For each vehicle class, we choose to estimate the initial OD demand vector corresponding to the selected PM peak period (5 pm -- 7 pm) of April 2012, by leveraging $|\scrK| = 120$ samples of the flow vector.  We take the average of these 120 flow vectors so as to obtain the observed flow vector $\bx^*$.
	
	

\end{itemize}

	
	\section*{Acknowledgments}
	
	The authors would like to thank the Boston Region MPO, and Scott Peterson in particular, for supplying the EMA data and providing us invaluable clarifications throughout our work. 
	
	

	\clearpage
	
%	\section*{References}
	
%	\bibliographystyle{IEEEtran}
	\bibliographystyle{elsarticle-num}
	%\bibliography{bib1}

\bibliography{/home/yannisp/Private/bib/abbrev,/home/yannisp/Private/bib/IEEEabrv,/home/yannisp/Private/bib/communications,/home/yannisp/Private/bib/my,/home/yannisp/Private/bib/optimization,/home/yannisp/Private/bib/stochastics,/home/yannisp/Private/bib/various,/home/yannisp/Private/bib/bio,/home/yannisp/Private/bib/statistics,/home/yannisp/Private/bib/control,/home/yannisp/Private/bib/transportation,bib1}
	
	%\bibliography{C:/Users/Yannis/Documents/Private/bib/IEEEabrv,C:/Users/Yannis/Documents/Private/bib/abbrev,C:/Users/Yannis/Documents/Private/bib/communications,C:/Users/Yannis/Documents/Private/bib/my,C:/Users/Yannis/Documents/Private/bib/optimization,C:/Users/Yannis/Documents/Private/bib/stochastics,C:/Users/Yannis/Documents/Private/bib/control,C:/Users/Yannis/Documents/Private/bib/statistics,bib1}
	
	
	
\end{document}              
